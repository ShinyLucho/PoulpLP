\documentclass[12pt,a4paper,oneside,dvipsnames,table,svgnames,skins,theorems]{report}
\usepackage{ProfCollege}
\usepackage[tp]{tmbasev2} %pour le moment placé dans la racine du .tex ... !


%%\graphicspath{{C:/Users/Thom/Documents/latex nas/images/}} %chemin du fichier image
%\graphicspath{{C:/Users/Thomas/latex nas/images/}} %chemin du fichier image

\ordi{fixe} %pour fichier image
\matierechap{divers} %m pour maths, pc pour physique chimie, rien pour vierge
\version{1} %1 : PROF   2: ELEVE sans corrigé  3: corrigé des exos sans le cours
\setcounter{chapter}{0} %compteur de chapitre
\author{Rentrée 2}



\begin{document}

\chapter{Verbes de Consignes}

\begin{objectifs}

En groupe de 3 élèves.\\
\textbf{Partie 1} :
\begin{itemize}
\item Lire le recto des flashcard.
\item Lire le verso des flashcard.
\item Associer un recto à son verso pour reformer la carte.
\end{itemize}
\textbf{Partie 2} :
\begin{itemize}
\item En vous aidant des flashcard, Écrire dans la marge pour chaque consigne ci-dessous un verbe d'action qui aurait pu être donné en consigne par l'enseignant.
\end{itemize}

\end{objectifs}


Réponses obtenues dans des copies (certaines sont inventées pour l'exercice, d'autres non !).
\vspace{0.2cm}

\begin{enumerate}
\item Il y a 4 coupables potentiels, parmi ces 4 coupables l'un est un homme droitier. Le meurtrier est gaucher. On peut donc conclure que le meurtrier est une femme.
\item La relation qui donne la vitesse en fonction de $a$ et du temps s'écrit  : $v=\dfrac{0.25a}{t}$
\item On observe que la température augmente jusqu'à 11h du matin puis diminue ensuite régulièrement jusqu'à 15h.
\item Le poids idéal d'un enfant de 6 mois est de \SI{4.2}{\kilo\gram} d'après le graphique.
\item Les deux valeurs pour $x$ sont $x_1=100$ ou $x_2=-110$. Une vitesse n'étant pas négative, la réponse est $x_1=\SI{100}{\kilo\meter\per\hour}$.
\item Le poids est : $P=9.81 \times 100 = \SI{981}{\newton}$.
\item On peut représenter la situation par la figure ci-dessous : 
\begin{center}
\begin{Geometrie}[TypeTrace="MainLevee"]
u:=7mm;
pair A,B,C;
A=u*(1,1);
B-A=u*(5,1);
C=rotation(B,A,60);
marque_s:=marque_s/3;
trace Codelongueur(A,B,B,C,C,A,2);
trace polygone(A,B,C);
trace cotationmil(A,C,5mm,20,btex 6 cm etex);
label.llft(btex A etex,A);
label.rt(btex B etex,B);
label.top(btex C etex,C);
\end{Geometrie}
\end{center}
\item Dans un triangle on a le grand côté au carré égal à la somme des carrés des petits côtés : ceci démontre que le triangle est rectangle (réciproque de Pythagore).
\item Les droites $(d)$ et $(AB)$ sont parallèles.
\end{enumerate}





\newpage


%\Cartes[Loop=false,Theme=Verbe, ThemeSol=Action,Trame]{%
%Déterminer lalongueur manquante. On détaillera la
%démarche./\Pythagore[Entier,Exact]{ABC}{3}{4}{}%
%§test/\Pythagore[Exact]{CBA}{5}{12}{}%
%§test/\Pythagore[Exact]{CBA}{5}{12}{}%
%
%}



\Cartes[Loop=false,Trame,Theme=Verbes,ThemeSol=Action]{%
\begin{center}
{\huge CALCULER}
\end{center}/ Déterminer un résultat à l'aide d'une ou plusieurs opérations effectuées à la main ou à l'aide d'un outil numérique.%
§ \begin{center}
{\huge DÉCRIRE}
\end{center}/ Donner les propriétés de quelque chose par observation. %
§ \begin{center}
{\LARGE DÉMONTRER}
\end{center} /Fournir une preuve pour un résultat que l'on cherchait à obtenir. La preuve pourra être sous diverses formes (calculs, graphique...)%
§ \begin{center}
{\huge EN DÉDUIRE}
\end{center}/ Utiliser un résultat déjà obtenu pour arriver à une conclusion.%
§ \begin{center}
{\LARGE REPRÉSENTER}
\end{center} / Réaliser un schéma ou une autre figure pour expliquer un exemple, un concept.%
§ \begin{center}
{\huge EXPRIMER}
\end{center} / Écrire une formule, un calcul, en utilisant des lettres qui remplacent des grandeurs et qui sont reliées entre elles. %
§ \begin{center}
{\huge ÉMETTRE} \\
 une conjecture\end{center}/Formuler une hypothèse, donner une idée qui devront être expliqués, démontrés.  %
§ \begin{center}
{\LARGE DÉTERMINER} \\
 graphiquement \end{center}/Utiliser un graphique pour trouver la réponse à une question en réalisant une lecture sur ce graphique. %
§ \begin{center}
{\LARGE INTERPRÉTER}
\end{center} /A partir d'un résultat obtenu, le relier à une situation, lui donner du sens. Mise en contexte. %
}


\end{document}