\documentclass[a4paper, 8pt]{article}
\usepackage[latin1]{inputenc}
\usepackage[T1]{fontenc}
\usepackage[francais]{babel}
\usepackage{entete}
\usepackage{noitemsep}
\usepackage{euscript} 
\usepackage{amsmath,amssymb,amsfonts,amsthm}
\usepackage{graphicx,graphics,epsfig,subfigure,color}
\usepackage{url}
%\usepackage{algorithm2e}
\usepackage{multicol}
\usepackage{a4wide}
\usepackage{latexsym}
\usepackage{verbatim}
\setlength{\textheight}{23.5cm}
\setlength{\topmargin}{-1cm}
\setlength{\textwidth}{155mm}
\setlength{\oddsidemargin}{2mm}

%\renewcommand{\baselinestretch}{0.85}

%\input{macroAlgo}
%\dontprintsemicolon

\setlength{\parindent}{0pt}  %%suppression indentation


\begin{document}
\selectlanguage{francais}
\author{D.~Fourer, L. Lagon}
\newcommand{\universityname}{IUT d'\'Evry Val d'Essonne}
\newcommand{\deptname}{D\'epartement TC (S3)}
\newcommand{\years}{2023-2024}

%------------------- TITRE -----------------------------------------
\date{Septembre 2023} 
\TDHead{\universityname}{\deptname}{R3.12, RCN3 \years}{\large TD1: Fonctions avanc\'ees d'un tableur}
%\TDHead{DUT TC}{}{\large TIC3: Fonctions avanc\'ees d'un tableur}
%-------------------------------------------------------------------
%\underline{Objectifs:} Ma\^itriser un logiciel de tableur

\section{Mise en place de l'environnement de travail} %% 1h, 1h30

%% import / export csv, maitrise des formats Libreoffice/Excel
%% Mise en page dans l'environnement excel

%% import d'un fichier csv:   donnee / import fichier txt
\exost Cr\'eez un r\'epertoire sur le bureau portant le nom \verb? TD1_NOM_PRENOM ? Vous y d\'eposerez le contenu
de l'archive contenant les fichiers du TD disponible sur \url{http://fourer.fr/Ens/2324/RCN3/td_excel1.zip} que vous prendrez 
soin de d\'ecompresser.
%Assurez vous que vous parvenez \`a ouvrir le fichier.

\exost Ouvrez le fichier \verb? classe.csv ? avec un \'editeur de texte (e.g. Notepad). \`A quoi ce format
correspond-il? Quels sont les caract\`eres qui permettent de dissocier chaque champ de valeur?

\exost Ouvrez le fichier \verb? classe.csv ?  avec le logiciel tableur de votre choix (Libreoffice Calc ou Microsoft Excel)
et v\'erifiez que le contenu s'affiche correctement. Vous devrez d'abord d\'efinir le format du fichier csv permettant de dissocier 
chaque cellule (s\'eparateurs).

\exost Maintenant ouvrez le fichier \verb? coefficients.csv ? et importez son contenu dans une nouvelle feuille de calcul.
Vous prendrez le soin de donner un nom \`a chaque feuille de calcul (e.g. ``notes \'etudiants'' et ``coefficients des mati\`eres'').

\exost Effectuez un tri des lignes par ordre alphab\'etique des noms de famille. %(ordre lexicographique croissant)
Assurez vous que la premi\`ere ligne contenant les titres est inchang\'ee et que chaque nom conserve bien les informations correspondantes.
Vous pourrez comparer le r\'esultat obtenu avec le contenu du fichier csv d'origine.

\exost Mettez en page vos feuilles de calcul excel en ins\'erant des lignes de s\'eparations et en agrandissant les titres.
V\'erifiez le format de chaque cellule et en particulier les champs correspondant aux dates de naissance.
Figez la premi\`ere ligne contenant les titres des colonnes afin que celle-ci reste visible m\^eme lorsque l'\'ecran d\'efile vers le bas.

\exost Enregistrez votre travail dans un nouveau fichier \verb? classe.xlsx ?. Vous vous assurerez que le fichier porte bien l'extension
XLSX et qu'il contient bien l'ensemble des modifications effectu\'ees quand vous l'ouvrez avec Libreoffice Calc ou Microsoft Excel.

\exost Essayez d'ouvrir le m\^eme fichier avec Libreoffice Calc et avec Microsoft Excel. Remarquez vous des diff\'erences?

\clearpage

\section{Formules math\'ematiques} %%duree 30min, 1h maxi

\exost Ins\'erez une nouvelle colonne avant la colonne \verb? NOM ? que vous nommerez \verb? ID ? (identifiant).
Affectez un num\'ero distinct \`a chaque \'etudiant. Vous commencerez par affecter un nombre au premier \'etudiant
puis vous appliquerez la formule $=x+1$ \`a l'\'etudiant suivant ($x$ repr\'esente le code de la cellule associ\'e \`a l'ID pr\'ec\'edent).
En utilisant la fonction copier-coller, vous appliquerez cette formule \`a l'ensemble des \'etudiants.

\exost Ins\'erez une nouvelle colonne que vous appellerez \verb? AGE ?. En utilisant la fonction \verb?DATEDIF()?,
calculez l'age de chaque \'etudiant \`a partir de sa date de naissance. Si vous le souhaitez, vous pourrez utiliser la fonction
\verb?MAINTENANT()? qui retourne la date du jour.

\exost Ins\'erez une nouvelle colonne apr\`es la derni\`ere note que vous appellerez \verb? MOYENNE ?. 
Vous ins\'ererez la moyenne pour l'ensemble des mati\`eres de chaque \'el\`eve dans cette colonne en utilisant 
la fonction \verb? =MOYENNE(x:y) ? o\`u $x$ et $y$ sont respectivement les codes de la premi\`ere et de la derni\`ere 
cellule prise en compte dans le calcul.

%%duree 30min, 1h maxi
\exost Ajoutez une nouvelle colonne \verb? MEDIANE ? puis calculez la note m\'ediane pour chaque \'etudiant. 
Vous pourrez rechercher la fonction correspondante en utilisant l'assistant (bouton ``Fx'').

\exost Ajoutez une nouvelle colonne \verb? MOYENNE PONDEREE ? Pour cela vous impl\'ementerez la formule suivante:

\begin{equation}
 \bar{x} = \frac{\sum_i c_i x_i}{\sum_i c_i} = \frac{c_1 x_1 + c_2 x_2 + ...}{c_1 + c_2 + ...}
\end{equation}

o\`u les $c_i$ correspondent aux coefficients de chaque mati\`ere que vous r\'ecup\'ererez dans la seconde feuille de calcul et
les $x_i$ correspondent aux notes obtenues pour chaque mati\`ere $i$.

\exost Utilisez la fonction \verb? RANG() ? pour afficher le classement de chaque \'etudiant en fonction de sa moyenne pond\'er\'ee
dans une nouvelle colonne que vous ajouterez.


\exost Ajoutez une nouvelle ligne en bas de page que vous intitulerez: ``Total des effectifs''.
Vous calculerez pour l'ensemble des \'etudiants la moyenne, la m\'ediane et l'\'ecart type de chaque mati\`ere ainsi que de la moyenne pond\'er\'ee.
Pour cela vous pourrez appliquer des formules matricielles pour l'ensemble des notes.

%\section{Graphiques et } %%duree 30min, 1h maxi

\textbf{BONUS:} Ins\'erez un graphique en permettant de visualiser la r\'epartition F/G, l'\^age et la r\'epartition des notes.

\end{document}

% End Of File

