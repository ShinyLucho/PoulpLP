\documentclass[12pt]{article}
\usepackage[utf8]{inputenc}
\usepackage[francais]{babel}
\usepackage{entete}
\usepackage{hyperref}
\usepackage{enumitem}
\usepackage{noitemsep}
\usepackage{euscript} 
\usepackage{amsmath,amssymb,amsfonts,amsthm}
\usepackage{graphicx,graphics,epsfig,subfigure,color}
\usepackage{url}
%\usepackage{algorithm2e}
\usepackage{multicol}
\usepackage{a4wide}
\usepackage{latexsym}
\usepackage{verbatim}
\setlength{\textheight}{23.6cm}
\setlength{\topmargin}{-2cm}
\setlength{\textwidth}{175mm}
\setlength{\oddsidemargin}{1.5mm}
\usepackage{pdfpages}
%\renewcommand{\baselinestretch}{0.85}

\pagenumbering{gobble}  %% remove page number

%\input{macroAlgo}
%\dontprintsemicolon

\setlength{\parindent}{0pt}  %%suppression indentation

\newif\ifcorrection
\correctiontrue   %% With correction
\correctionfalse   %% Reviewer's version


\begin{document}
\selectlanguage{francais}
\author{L. Lagon, D. Fourer, S. Compagnon}
\newcommand{\universityname}{IUT TCJ}
\newcommand{\deptname}{RCN 1}
\newcommand{\years}{2024-2025}

%------------------- TITRE -----------------------------------------
\date{Septembre 2021} 
\TDHead{\universityname}{\deptname}{ \years}{\large TP Noté : RCN 1}

\begin{itemize}
    \item Le travail devra impérativement être envoyé en fin de séance à dominique.fourer@univ-evry.fr
    \item 1 point porte sur la présentation et le respect des consignes 
    \item Nous rappelons que l'utilisation d'IA dans le cadre de cet examen est interdite. La consultation des documentations Excel est cependant autorisée. 
    \item L'ouverture de tout fichier Word ou Excel non relatif à l'examen entraînera la note de 0.
\end{itemize}

\section{Mettre en forme avec Word (5 points)}

\textit{Dans cette partie de l'examen, le travail est à réaliser dans le fichier "Word.doc"}

\begin{enumerate}
    \item Affecter aux différentes parties (en gras) un style de titre 1. Affecter aux différentes sous-parties (en italique) un style de titre 2.
    \item Définir le style titre 1 en gras, taille 14, souligné et assigner lui une couleur au choix. Numéroter en chiffre.
    \item Définir le style titre 2 en gras, taille 12, avec un autre couleur au choix. Lui ajouter un retrait vers la gauche. Numéroter en lettres.
    \item Ajouter des numéros de pages centrés 
    \item Insérer un sommaire automatique
\end{enumerate}

\section{Automatiser avec Excel (14 points)}
\textit{Dans cette partie de l'examen, le travail est à réaliser dans le fichier "Word.doc"}

Vous êtes chargé de calculer les salaires des différents employés de votre entreprise. Vous disposez des données suivantes :

\begin{itemize}[itemsep=0.1cm]
    \item L'employé dispose d'un \textbf{fixe} de 800 €
    \item Une \textbf{prime de fidélisation} représentant 20 € par années passées dans l'entreprise est accordée. Par exemple, si une personne à 10 ans d'ancienneté, elle reçoit 200€ de prime de fidélisation.
    \item Une \textbf{commission} représentant 1,5\% du chiffre d'affaire est accordée.
    \item Une \textbf{prime compétitivité} de 10 000 € est à partager aux employés au pro-rata de leur participation au CA. Par exemple, si un vendeur totalise 5\% des ventes du mois, il touchera 5\% de 10 000 €. 
    \item Une \textbf{prime d'assiduité} de 50 € si l'employé a moins de 7 jours d'absences et qu'il a un CA supérieur à 5000 €. 
\end{itemize}

\newpage

\begin{enumerate}[itemsep=0.1cm]
    \item (1 point) Pour chacun des salarié, ajouter un salaire fixe de 800 €.
    \item (2 points) A l'aide de la fonction DATEDIF, afficher dans une colonne le nombre d'années qu'un employé à passé dans l'entreprise. Puis, calculer le montant de sa prime de fidélisation.
    \item (2 point) Calculer la commission versée à chaque vendeur.
    \item (3 points) Calculer le total du CA du mois réalisé par la totalité des vendeur. Puis, calculer le \% des ventes totales réalisé par chaque vendeur. Enfin, calculer la prime compétitivité. \textbf{Remarque :} Il est tout à fait possible de réaliser ces 2 dernières opérations sur une seule colonne. 
    \item (2 points) Dans une colonne, afficher "Trop de retards" si le vendeur cumule plus de 7 jours de retard. Puis, à l'aide d'une mise en forme conditionnelle, afficher cette cellule en rouge.
    \item (2 points) Calculer la prime d'assiduité 
    \item (2 points) Calculer le salaire total des vendeurs 
\end{enumerate}

\begin{figure}
    \centering
    \includegraphics[width=1\linewidth]{image.png}
    \caption{Exemple de réalisation (Disponible en PJ)}
    \label{fig:enter-label}
\end{figure}
\end{document}



% End Of File

