\documentclass[a4paper]{article}
\usepackage[latin1]{inputenc}
\usepackage[T1]{fontenc}
\usepackage[francais]{babel}
\usepackage{entete}
\usepackage{noitemsep}
\usepackage{euscript} 
\usepackage{amsmath,amssymb,amsfonts,amsthm}
\usepackage{graphicx,graphics,epsfig,subfigure,color}
\usepackage{url}
\usepackage{algorithm2e}
\usepackage{multicol}
\usepackage{a4wide}
\usepackage{latexsym}
\usepackage{verbatim}
\setlength{\textheight}{23.5cm}
\setlength{\topmargin}{-1cm}
\setlength{\textwidth}{155mm}
\setlength{\oddsidemargin}{2mm}

%\renewcommand{\baselinestretch}{0.85}

%\input{macroAlgo}
\dontprintsemicolon

\setlength{\parindent}{0pt}  %%suppression indentation

%% ACRONYM DEFINITIONS

\newcommand{\lcx}[1]{\MakeLowercase{#1}}
%\renewcommand{\log}[0]{\text{ln}}
\renewcommand{\iint}[0]{\int\!\!\!\!\!\int}
\renewcommand{\iiint}[0]{\int\!\!\!\!\!\int\!\!\!\!\!\int}
\renewcommand{\iiiint}[0]{\int\!\!\!\!\!\int\!\!\!\!\!\int\!\!\!\!\!\int}
%% expression

%% Latin abrev.
\newcommand{\etc}[0]{\textit{etc} }
\newcommand{\etal}[0]{\textit{et al.} }
\newcommand{\eg}[0]{\textit{e.g.} }
\newcommand{\ie}[0]{\textit{i.e.} }
\newcommand{\cf}[0]{\textit{cf.} }

%% Mathematical coordinates
\newcommand{\meq}[0]{\!\!=\!\!}
\newcommand{\TW}[0]{(\tau, \Omega)}
\newcommand{\tw}[0]{(t, \omega)}
\newcommand{\tv}[0]{(t, \Omega)}
\newcommand{\ts}[0]{(t, s)}
\newcommand{\sw}[0]{(s, \omega)}
\newcommand{\zw}[0]{(z, \omega)}
\newcommand{\nm}[0]{[n,m]}
\newcommand{\uv}[0]{[u,v]}

%% Mathematical constants
\newcommand{\ww}[0]{|\omega|}   %% 1
\newcommand{\www}[0]{|\omega|}  %% []
\newcommand{\te}[0]{T_s}
\newcommand{\fs}[0]{F_s}
\newcommand{\ee}{\,\ensuremath{\mathbf{e}}}

%% Mathematical functions names
\renewcommand{\Re}[0]{\text{Re}}
\renewcommand{\Im}[0]{\text{Im}}
\newcommand{\ri}[0]{\text{Ri}}
\newcommand{\wvd}[0]{\text{WV}}
\newcommand{\cwt}[0]{\text{CW}}
\newcommand{\mwt}[0]{\text{MW}}
\newcommand{\sst}[0]{\text{SST}}
\newcommand{\swt}[0]{\text{SWT}}
\newcommand{\cre}[0]{\text{CRE}}
\newcommand{\card}[1]{\text{Card}(#1)}

%% Mathematical functions names
\newcommand{\rst}[0]{\text{RST}}
\newcommand{\vsst}[0]{\text{VSST}}
\newcommand{\osst}[0]{\text{OSST}}
\newcommand{\lmsst}[0]{\text{LMSST}}

%% Mathematical variables names
\newcommand{\TT}[0]{\mathcal{T}}
\newcommand{\DD}[0]{\mathcal{D}}
%% EOF

\newif\ifcorrection
%\correctiontrue   %% avec correction
\correctionfalse   %% sans correction

\newcommand{\universityname}{IUT d'\'Evry Val d'Essonne}
\newcommand{\deptname}{D\'epartement TC (S3)}
\newcommand{\years}{2023-2024}

\begin{document}
\selectlanguage{francais}
\author{D. Fourer, L. Lagon}
%------------------- TITRE -----------------------------------------
\date{Septembre 2022} 
\TDHead{\universityname}{\deptname}{R3.07 TQR3, \years}{\large TD1: D\'enombrement et probabilit\'es}
%-------------------------------------------------------------------


\underline{Rappels:} 

Factorielle: $n! = 1\times 2 \times \cdots \times n$\\
Combinaisons: $C_n^k = \frac{n!}{k! (n-k)!}$\\
Arrangements: $A_n^k = \frac{n!}{(n-k)!}$\\
Permutations: $P_n = A_n^n = n!$

% \begin{table}
%  \begin{tabular}{l l l}
%  \hline
%  avec ordre 	& avec remise 	& # de cas \\
%  \hline
%  \hline
%  non		& non		& $C_n^k$ \\
%  \hline
%  oui		& non		& $A_n^k$ \\
%  \hline
%  non		& oui		& $C_{n+k-1}^k$ \\
%  \hline
%  oui		& oui		& $C_{n+k-1}^k$ \\
% \end{tabular}
% \caption{D\'enombrement pour $k$ tirages de $n$ \'el\'ements distincts.}
% \end{table}

\exost
On dispose de l'alphabet latin sans accent de 26 lettres.
\begin{enumerate}
 \item Combien de mots diff\'erents peut on \'ecrire en utilisant toutes les lettres du mot \og{}facile\fg{}?
   \ifcorrection
 \textcolor{red}{
 ``Facile'' s'\'ecrit avec 6 lettres, donc il existe $P_6=6!=720$ permutations possibles.
 }
 \fi
 \item Combien de mots diff\'erents de exactement 6 lettres peut on \'ecrire avec tout l'alphabet? (on autorise les r\'ep\'etitions)
   \ifcorrection
 \textcolor{red}{
 On consid\`ere un tirage ordonn\'e de 6 \'el\'ements avec remise, soit $26^6 = 308~915~776$ mots possibles.
 }
 \fi
 \item Combien de mots diff\'erents de exactement 6 lettres peut on \'ecrire sans r\'ep\'etition?
   \ifcorrection
 \textcolor{red}{
 On consid\`ere un tirage ordonn\'e de 6 \'el\'ements sans remise, soit $A_{26}^6 = \frac{26!}{(26-6)!} = 165~765~600$ mots possibles.
 }
 \fi
 \item Combien de mots diff\'erents de 5 lettres ou moins peut on \'ecrire sans r\'ep\'etition (dans le m\^eme mot)?
   \ifcorrection
 \textcolor{red}{
 On consid\`ere la somme de tous les tirages ordonn\'es de 5 \'el\'ements ou moins sans remise, soit $A_{26}^1 + A_{26}^2 + A_{26}^3 + A_{26}^4 + A_{26}^5 = 26 + 26\times 25 + 26\times 25\times 24 + 26\times 25\times 24\times 23 + A_{26}^5 = 8~268~676$ mots possibles.
 }
 \fi
 \item On tire d\'esormais au hasard 3 lettres non ordonn\'e sans r\'ep\'etition. Quelle est la probabilit\'e de d'obtenir le mot \og{}IUT\fg{} si tous les tirages sont \'equiprobables?
  \ifcorrection
  \textcolor{red}{
   On consid\`ere le nombre d'arrangements de 3 lettres parmi 26 lettres: $\card{\Omega}=A_{26}^3=\frac{26!}{(26-3)!} = 15~600$.\\
   Si on note $A$ l'\'ev\'enement $\{\text{``obtenir le mot IUT''} \}$, alors $P(A) = \frac{1}{15~600}$
  }
  \fi
\end{enumerate}



\exost 
On propose \`a un examen un questionnaire \`a choix multiples (QCM) avec 8 questions.
Pour chaque question, il y a 3 r\'eponses possibles dont une seule est correcte.
Le candidat d\'ecide de r\'epondre au hasard en ne cochant qu'une seule case \`a chaque question.
\begin{enumerate}
 \item Combien il y a-t-il de fa\c{c}ons diff\'erentes de remplir le questionnaire?
  \ifcorrection
 \textcolor{red}{
  On applique la formule d'un choix de $k=8$ \'el\'ements ordonn\'es parmis 3 avec remise:
  Il y a donc $\card{\Omega} = 3^8= 6561$ choix possibles.
 }
 \fi
 \item Combien de grilles diff\'erentes ne comportent qu'une seule r\'eponse fausse.
  \ifcorrection
 \textcolor{red}{
 Pour chaque question il y a 2 r\'eponses fausses. En supposant, toutes les autres r\'eponses justes, on fait la somme des r\'eponses fausses
 contenues dans le QCM, soit $2\times 8 = 16$ grilles diff\'erentes avec une seule faute.
 }
 \else
 \fi
 \item Combien de grilles diff\'erentes possibles sont enti\`erement fausses?
 \ifcorrection
 \textcolor{red}{
 Pour chaque question il y a 2 r\'eponses fausses. En supposant, toutes les autres r\'eponses justes, on fait la somme des r\'eponses fausses
 soit $2^8 = 256$ grilles diff\'erentes enti\`rement fausses.
 }
  \else
 \fi
  \item Combien de grilles diff\'erentes possibles ont au moins une bonne r\'eponse?
  \ifcorrection
 \textcolor{red}{
 On note $A$ l'\'ev\'enement $\{\text{``La grille est enti\`erement fausse''}\}$, donc $\bar{A} = \{ \text{``La grille contient une bonne r\'eponse''}\}$.
 Comme on sait que $\card{A}=256$ (cf. r\'eponse pr\'ec\'edente), alors $\card{\bar{A}} = 6561-256 = 6305$.
 }
 \else
 \fi
\end{enumerate}



%% Denombrement
\exost
Les membres d'une association de 20 personnes (13 hommes et 7 femmes) souhaitent constituer un bureau de 3 personnes (un(e) pr\'esident(e), un(e) tr\'esorier(e)
et un(e) secr\'etaire).
\begin{enumerate}
 \item Combien de bureaux (groupe de 3 personnes) diff\'erents peuvent \^etre constitu\'es \`a partir de ces 20 personnes?
 \ifcorrection
 \textcolor{red}{
  On applique la formule d'un tirage ordonn\'e de 3 \'el\'ements parmis 20 $A_{20}^3 = \frac{20!}{17!} = 20 \times 19 \times 18 = 6840$.
  }
  \else
 \fi
 \item Combien de bureaux diff\'erents ayant une femme pr\'esidente peuvent \^etre constitu\'es?
  \ifcorrection
  \textcolor{red}{
  On applique la formule d'un tirage ordonn\'e de 2 \'el\'ements parmis 19: $A_{19}^2 = \frac{19!}{17!} = 19 \times 18  = 342$.
  Comme il y a 7 femmes, alors au total il y a $7 \times 342 = 2394$ bureaux possibles avec 1 femme pr\'esidente.
  (et donc $13 \times 342 = 4446$ bureaux avec 1 homme pr\'esident). On v\'erifie que $2394 + 4446 = 6840$ correspond au total.  
  }
 \fi
 \item En supposant \'equiprobable le choix de chaque candidat. Quelle est la probabilit\'e pour que le bureau soit compos\'e d'au moins une femme?
  \ifcorrection
  \textcolor{red}{
  %On note $\Omega$ l'ensemble des bureaux possibles tel que $\card{\Omega} = C_{20}^3 = 1140$.
  -On note $F$ l'\'ev\'enement, $\{\text{``une femme est dans le bureau''}\}$ et son compl\'ementaire $\bar{F}=\{\text{``Il n'y a pas de femme dans le bureau''} \}$.
  Le tirage de chaque candidat est ind\'ependant donc:
  $P(\bar{F}) = \frac{13}{20} \frac{12}{19} \frac{11}{18} = \frac{1716}{6840} \approx 0,25$.
  Donc $P(F) = 1 - P(\bar{F}) \approx 0,75$.\\
  -Autre solution: Il y a $A_{13}^3$ arrangements de 3 hommes, donc $P(\bar{F}) = \frac{A_{13}^3}{A_{20}^3} = \frac{\frac{13!}{(13-3)!}}{\frac{20!}{(20-3)!}}=\frac{13\times 12 \times 11}{20\times 19 \times 18}\approx 0,25$\\
  -Autre solution: On \'enum\`ere toutes les combinaisons (avec au moins 1F): 3F: $C_7^3$, 2F1M: $C_7^2 C_{13}^1$, 1F2M: $C_7^1C_{13}^2$, donc
  $P(F) = \frac{C_7^3 + (C_7^2C_{13}^1) + (C_7^1C_{13}^2)}{C_{20}^3} = \frac{854}{1140} \approx 0,75$.
  }
  \else
 \fi
\end{enumerate}

\exost 
Un sac contient 13 boules noires et 2 boules rouges. Combien faut il en tirer simultan\'ement
pour que la probabilit\'e d'obtenir au minimum une boule rouge soit sup\'erieure \`a $\frac{6}{7}$?

\ifcorrection
  \textcolor{red}{
  On note $x$ l'inconnue (le nombre de boules tir\'ees) et $\bar{A}$ l'\'ev\'enement compl\'ementaire $\{\text{``on ne tire aucune boule rouge''}\}$.
  On a $\card{\bar{A}} = C_{13}^x$ et $\card{\Omega} = C_{15}^x$.\\
  Il faut alors r\'esoudre l'in\'equation:
  $P(A) = 1 - \frac{C_{13}^x}{C_{15}^x} > \frac{6}{7}$.
  On a donc:
  \begin{align}
   \frac{1}{7} 			&> \frac{\frac{13!}{x!(13-x)!}}{\frac{15!}{x!(15-x)!}} = \frac{13!(15-x)!}{15!(13-x)!} = \frac{(15-x)(14-x)}{15\times 14}\\
   \frac{15 \times 14}{7}	&> x^2 - 29x + 210\\
   0 				&> x^2 - 29x + 180
  \end{align}
  On pose $\Delta = b^2 - 4 ac$ ($b=-29$, $a=1$ et $c=180$) donc $\Delta = 121 = 11^2$.\\
  On a donc (la plus petite solution) $x>\frac{-b-\sqrt{\Delta}}{2a} = \frac{29-11}{2} = 9$.\\
  L'autre solution $x_2>\frac{-b+\sqrt{\Delta}}{2a} = 20$ est bien plus grande que 9 mais sans int\'er\^et car on cherche le nombre
  minimum de tirage.
  }
  \else
 \fi
 
\end{document}

% End Of File

