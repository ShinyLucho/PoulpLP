\documentclass[a4paper]{article}
\usepackage[latin1]{inputenc}
\usepackage[T1]{fontenc}
\usepackage[francais]{babel}
\usepackage{entete}
\usepackage{noitemsep}
\usepackage{euscript} 
\usepackage{amsmath,amssymb,amsfonts,amsthm}
\usepackage{graphicx,graphics,epsfig,subfigure,color}
\usepackage{url}
\usepackage{algorithm2e}
\usepackage{multicol}
\usepackage{a4wide}
\usepackage{latexsym}
\usepackage{verbatim}
\setlength{\textheight}{23.5cm}
\setlength{\topmargin}{-1cm}
\setlength{\textwidth}{155mm}
\setlength{\oddsidemargin}{2mm}

%\renewcommand{\baselinestretch}{0.85}

%\input{macroAlgo}
\dontprintsemicolon

\setlength{\parindent}{0pt}  %%suppression indentation

%% ACRONYM DEFINITIONS

\newcommand{\lcx}[1]{\MakeLowercase{#1}}
%\renewcommand{\log}[0]{\text{ln}}
\renewcommand{\iint}[0]{\int\!\!\!\!\!\int}
\renewcommand{\iiint}[0]{\int\!\!\!\!\!\int\!\!\!\!\!\int}
\renewcommand{\iiiint}[0]{\int\!\!\!\!\!\int\!\!\!\!\!\int\!\!\!\!\!\int}
%% expression

%% Latin abrev.
\newcommand{\etc}[0]{\textit{etc} }
\newcommand{\etal}[0]{\textit{et al.} }
\newcommand{\eg}[0]{\textit{e.g.} }
\newcommand{\ie}[0]{\textit{i.e.} }
\newcommand{\cf}[0]{\textit{cf.} }

%% Mathematical coordinates
\newcommand{\meq}[0]{\!\!=\!\!}
\newcommand{\TW}[0]{(\tau, \Omega)}
\newcommand{\tw}[0]{(t, \omega)}
\newcommand{\tv}[0]{(t, \Omega)}
\newcommand{\ts}[0]{(t, s)}
\newcommand{\sw}[0]{(s, \omega)}
\newcommand{\zw}[0]{(z, \omega)}
\newcommand{\nm}[0]{[n,m]}
\newcommand{\uv}[0]{[u,v]}

%% Mathematical constants
\newcommand{\ww}[0]{|\omega|}   %% 1
\newcommand{\www}[0]{|\omega|}  %% []
\newcommand{\te}[0]{T_s}
\newcommand{\fs}[0]{F_s}
\newcommand{\ee}{\,\ensuremath{\mathbf{e}}}

%% Mathematical functions names
\renewcommand{\Re}[0]{\text{Re}}
\renewcommand{\Im}[0]{\text{Im}}
\newcommand{\ri}[0]{\text{Ri}}
\newcommand{\wvd}[0]{\text{WV}}
\newcommand{\cwt}[0]{\text{CW}}
\newcommand{\mwt}[0]{\text{MW}}
\newcommand{\sst}[0]{\text{SST}}
\newcommand{\swt}[0]{\text{SWT}}
\newcommand{\cre}[0]{\text{CRE}}
\newcommand{\card}[1]{\text{Card}(#1)}

%% Mathematical functions names
\newcommand{\rst}[0]{\text{RST}}
\newcommand{\vsst}[0]{\text{VSST}}
\newcommand{\osst}[0]{\text{OSST}}
\newcommand{\lmsst}[0]{\text{LMSST}}

%% Mathematical variables names
\newcommand{\TT}[0]{\mathcal{T}}
\newcommand{\DD}[0]{\mathcal{D}}
%% EOF

\newif\ifcorrection
%\correctiontrue   %% Final version
\correctionfalse   %% Reviewer's version

\newcommand{\universityname}{IUT d'\'Evry Val d'Essonne}
\newcommand{\deptname}{D\'epartement TC (S3)}
\newcommand{\years}{2023-2024}

\begin{document}
\selectlanguage{francais}
\author{D. Fourer, L. Lagon}
%------------------- TITRE -----------------------------------------
\date{Septembre 2023} 
\TDHead{\universityname}{\deptname}{R3.07 TQR3, \years}{\large TD4: Lois usuelles et statistiques}
%-------------------------------------------------------------------

\underline{Rappels:} 

\begin{itemize}
 \item Loi binomiale $X \sim \beta(n,p)$: $P(X=x) = C_n^x p^x(1-p)^{n-x}$
 \item Loi de Poisson $X \sim \mathcal{P}(\lambda)$: $P(X=x)=\frac{\ee^{-\lambda}\lambda^x}{x!} \quad ,\forall \lambda > 0$
 \item Esp\'erance: $E[X] = \sum_x x P(X=x)$
 \item Variance: $V[X] = E[(X-E[X])^2] = E[X^2] - E[X]^2$
% \item $P(a < X \leq b) = F_X(b)-F_X(a)$
\end{itemize}

~\par{}


\exost On jette successivement, cinq fois un d\'e. On s'int\'eresse \`a la 
v.a. $X$ qui repr\'esente le nombre de fois qu'un $3$ appara\^it.
\begin{enumerate}
 \item D\'efinir $X$ (seulement les ensembles de d\'epart et d'arriv\'ee).
 \ifcorrection
 \textcolor{red}{~\\
 $X: \{1,2,3,4,5,6\}^5 \rightarrow \{0,1,2,3,4,5\}$
 }
 \fi
 \item D\'efinir la loi de probabilit\'e de $X$ (avec ses param\`etres).
 \ifcorrection
 \textcolor{red}{~\\
 $X \sim \beta(n,p)$ avec $n=5$ et $p=\frac{1}{6}$.
 }
 \fi
 \item Quelle est la probabilit\'e d'obtenir le nombre $3$ deux fois?
 \ifcorrection
 \textcolor{red}{~\\
 $P(X=2) = C_5^2 (\frac{1}{6})^2 (\frac{5}{6})^{3} = 0,1608$
 }
 \fi
 \item Quelle est la probabilit\'e de ne pas obtenir le nombre $3$?
 \ifcorrection
 \textcolor{red}{~\\
 $P(X=0) = C_5^0 (\frac{1}{6})^0 (\frac{5}{6})^{5} = 0,4019$
 }
 \fi
 \item Quelle est la probabilit\'e d'obtenir le nombre $3$ au moins une fois?
 \ifcorrection
 \textcolor{red}{~\\
 $\sum_{x=1}^5 P(X=x) = 1-P(X=0) = 0,5981$
 }
 \fi 
 \item Quelle est la probabilit\'e d'obtenir le nombre $3$ au moins $3$ fois?
 \ifcorrection
 \textcolor{red}{~\\
  $\sum_{x=3}^5 P(X=x) = 0,0322 + 0.0032 + 1,28 \times 10^{-4}$
 }
 \fi
 \item Quelle est l'esp\'erance de $X$?
  \ifcorrection
 \textcolor{red}{~\\
  $E[X] = np = \frac{5}{6} \approx 0,83$
 }
 \fi
\end{enumerate}


%% Loi binomiale
\exost On consid\`ere la v.a. $X$ qui mod\'elise le nombre de fois qu'un tireur \`a l'arc
atteint sa cible apr\`es $n$ tirs. Sachant que ce tireur a une probabilit\'e $p=\frac{5}{7}$ d'atteindre sa cible.
\begin{enumerate}
 \item D\'efinir la v.a. $X$.
  \ifcorrection
 \textcolor{red}{~\\
 $X: \Omega \rightarrow \{0,1,2,...,n\}$\\
 $X \sim \beta(n,p)$ avec $p=\frac{5}{7}$
 }
 \fi
 \item Quelle est la probabilit\'e que le tireur touche 3 fois sa cible apr\`es 5 tirs? M\^eme question apr\`es 10 tirs?
  \ifcorrection
  \textcolor{red}{~\\
  $X \sim \beta(5,\frac{5}{7}) \Rightarrow P(X=3) = C_5^3 (\frac{5}{7})^3 (\frac{2}{7})^{2}\approx 0,2975$\\
  $X \sim \beta(10,\frac{5}{7}) \Rightarrow P(X=3) = C_{10}^3 (\frac{5}{7})^3 (\frac{2}{7})^{7}\approx 0,0068$
 }
 \fi
 \item On suppose que le tireur gagne 1 euro \`a chaque fois qu'il touche sa cible. D\'efinir cette nouvelle v.a. $Y$ qui 
 d\'efinit le gain du tireur apr\`es un tir.
  \ifcorrection
  \textcolor{red}{~\\
  La nouvelle v.a $Y$ associe un gain de 1 euro \`a chaque fois qu'un tireur touche sa cible donc 
  on a $Y=X$.
  }
  \fi
 \item Calculer l'esp\'erance math\'ematique de $Y$ apr\`es 5 tirs.
   \ifcorrection
  \textcolor{red}{~\\
  $Y$ suit une loi binomiale de param\`etre $p=5/7$ et $n=5$ donc
  $E[Y] = E[X] = np = \frac{25}{7}\approx 3,5714$
  }
  \fi
 \item Calculer la variance de $Y$ apr\`es 5 tirs.
   \ifcorrection
  \textcolor{red}{~\\
  de m\^eme on a $V[Y] = V[X] = n p (1-p) = \frac{5\times 5 \times 2}{7} = 5 \frac{5 \times 2}{7\times 7} = \frac{50}{49} \approx 1,0204$
  }
  \fi
\end{enumerate}



%% Loi Poisson
\exost Une \'etude r\'ealis\'ee par un technicien a permis d'\'etablir que le nombre moyen
des arriv\'ees de pi\`eces \`a usiner \`a un certain poste est de 90 \`a l'heure.
En supposant que la v.a. $X$ qui compte le nombre d'arriv\'ees \`a la minute 
suit une loi de Poisson.
\begin{enumerate}
 \item D\'efinir la loi de probabilit\'e de $X$
 \item Quelle est la probabilit\'e qu'entre 10h52 et 10h53 il n'y ait aucune arriv\'ee?
 \item Quelle est la probabilit\'e que pendant une minute il y ait entre 2 et 5 arriv\'ees?
 \item Quelle est l'esp\'erance de $X$?
 \item Quelle est la variance de $X$?
\end{enumerate}



% \exost On se donne la fonction de r\'epartition $F_X(x)$ d'une v.a. $X$, repr\'esent\'ee par la table suivante:
% %
% \begin{tabular}{|l|l|l|l|l|l|l|l|l|l|l|l|l|}
% \hline
%  $x$		& 2 			& 3 			& 4 			& 5 			& 6 			& 7			& 8 			& 9 			& 10 			& 11 			& 12\\
% \hline
%   $F_X(x)$	& $\frac{1}{36}$	& $\frac{3}{36}$	& $\frac{6}{36}$	& $\frac{10}{36}$ 	& $\frac{15}{36}$	& $\frac{21}{36}$	& $\frac{26}{36}$	& $\frac{30}{36}$	& $\frac{33}{36}$	& $\frac{35}{36}$ 	& 1\\
% \hline
% \end{tabular}
% %
% \begin{enumerate}
%  \item Faire la repr\'esentation graphique de la fonction $x \mapsto F_X(x)$ (Vous pourrez prendre $\frac{1}{36}$ comme unit\'e sur l'axe des ordonn\'ees) 
%   \ifcorrection
%  \textcolor{red}{
%  % F = [1/36 3/36 6/36 10/36 15/36 21/36 26/36 30/36 33/36 35/36 1]
%  % bar(2:12, F)
%  % xlabel('x')
%  % ylabel('F_X')
%  \begin{figure}[!ht]
%  \centering\includegraphics[width=0.3\textwidth]{fonction_repartition}
%  \end{figure}
%  }
%  \fi
%  \item Calculer les probabilit\'es suivantes: $P(5 < X \leq 10)$, $P(X \leq 3)$, $P(X > 10)$.
%  \ifcorrection
%  \textcolor{red}{
%    $P(5 < X \leq 10) = F_X(10)-F_X(5) = \frac{33}{36} - \frac{10}{36} = \frac{23}{36} \approx 0,6389$\\
%    $P(X \leq 3) = F_X(3) = \frac{3}{36} = \frac{1}{12}$\\
%    $P(X > 10) = 1 - P(X \leq 10) = 1 - F_X(10) = 1 - \frac{33}{36} = \frac{3}{36} = \frac{1}{12}$
%   }
%  \fi
% %  \begin{enumerate}
% %   \item $P(5 < X \leq 10)$ ?
% %   \item $P(X \leq 3)$?
% %   \item $P(X > 10)$
% %  \end{enumerate}
%  \item A quelle exp\'erience al\'eatoire vue en cours peut \^etre associ\'ee la v.a. $X$?
%  \ifcorrection
%  \textcolor{red}{
%   Le lancer de 2 d\'es (lancer \'equiprobable), o\`u $X$ repr\'esente la somme des 2 valeurs.
%  }
%  \fi
%  \item Dans ce cas, qu'elle est la loi suivie par $X$? 
%    \ifcorrection
%   \textcolor{red}{
%    C'est la loi suivie par la somme de deux v.a. suivant une loi uniforme.\\
%    \begin{tabular}{|l|l|l|}
%    \hline
%     \'ev\'enement					& x	& $P(X=x)$ \\
%     \hline
%     \hline
%     $(1;1)$						& 2	& $\frac{1}{36}$ \\
%    \hline
%     $(1;2), (2;1)$					& 3	& $\frac{2}{36}$ \\
%     \hline
%     $(2;2), (1;3), (3;1)$				& 4	& $\frac{3}{36}$ \\
%     \hline
%     $(1;4), (2;3),(3;2),(4;1)$				& 5	& $\frac{4}{36}$ \\
%     \hline
%     $(1;5), (2;4), (3;3), (4;2), (5;1)$			& 6	& $\frac{5}{36}$ \\
%     \hline
%     $(1;6), (2;5), (3;4), (4;3), (5;2), (6;1)$		& 7	& $\frac{6}{36}$ \\
%     \hline
%     $(2;6), (3;5), (4;4), (5;3), (6;2)$			& 8	& $\frac{5}{36}$ \\
%     \hline
%     $(3;6), (4;5), (5;4), (6;3)$			& 9	& $\frac{4}{36}$ \\
%     \hline
%     $(4;6), (5;5), (6;4)$				& 10	& $\frac{3}{36}$ \\
%     \hline
%     $(5;6), (6;5)$					& 11	& $\frac{2}{36}$ \\
%     \hline
%     $(6;6)$						& 12	& $\frac{1}{36}$ \\
%     \hline
%    \end{tabular}
%   }
%   \fi
% \end{enumerate}
% 
% %% probleme concret
% \exost Dans une entreprise, on note $X$ et $Y$ les v.a. associ\'ees au nombre de ventes r\'ealis\'ees
% par chacun des deux commerciaux qui la composent.
% Nous disposons des lois $p_X$ et $p_Y$ donn\'ees par:\\
% \begin{tabular}{|l|l|l|l|l|l|}
% \hline
%  $x$		& 0    & 1    & 2    & 3      & 4 ou plus \\
%  \hline
%   $P(X=x)$	& 0,52 & 0,27 & 0,14 & 0.07   &  0 \\
%   \hline
%    $P(Y=x)$	& 0,46 & 0,31 & 0,16 & 0.07   &  0 \\
%   \hline
% \end{tabular}
% 
% \begin{enumerate}
%  \item D\'efinir la v.a. $Z=X+Y$ qui correspond au nombre total de contrats sign\'es par l'entreprise et sa loi de probabilit\'e $p_Z$
%     \ifcorrection
%   \textcolor{red}{
%   $Z: \{0;1;2;3 \} \times \{0;1;2;3 \} \rightarrow \{0;1;2;3;4;5;6 \}$\\
%   (faire un arbre)\\
%   \begin{tabular}{|l|l|l|l|l|l|l|l|l|}
%   \hline
%    z 	     & 0      & 1 	& 2 		& 3 	 & 4       & 5       & 6      & 7 et plus \\
%    \hline
%    $P(Z=z)$  & 0.2392 & 0.2854	& 0,2313	& 0.1552 &  0.0630 & 0.0210  & 0.0049 & 0 \\
%    \hline   
%   \end{tabular}
%   }
%   \fi
%  \item D\'efinir la v.a. $W=|X-Y|$ qui correspond \`a l'\'ecart du nombre de vente entre les 2 commerciaux et sa loi de probabilit\'e $p_Z$
%    \ifcorrection
%   \textcolor{red}{
%   $W: \{0;1;2;3 \} \times \{0;1;2;3 \} \rightarrow \{0;1;2;3\}$\\
%   (faire un arbre)\\
%    \begin{tabular}{|l|l|l|l|l|l|}
%    \hline
%     w 	      & 0       & 1 	& 2 		& 3 	 & 4 et plus \\
%     \hline
%     $P(W=w)$  &  0.3502 & 0.393 &   0.1882      & 0.0686 & 0 \\
%     \hline 
%    \end{tabular}
%   }
%   \fi
%  \item \'Ecrire le tableau des probabilit\'es des diff\'erentes valeurs du couple $(W,Z)$.
%    \ifcorrection
%   \textcolor{red}{
%   ~\\
%  \begin{tabular}{|l |l|l|l|l| l|l|l|l| l|l|l|l| l|l|l|l|}
%  \hline
%   X        & 0      & 0      & 0      & 0      & 1 & 1 & 1 & 1 &  2 & 2 & 2 & 2 & 3 & 3 & 3 & 3 \\
%   Y        & 0      & 1      & 2      & 3      & 0 & 1 & 2 & 3 &  0 & 1 & 2 & 3 & 0 & 1 & 2 & 3 \\
%   \hline
%   W        & 0      & 1      & 2      & 3      & 1 & 0 & 1 & 2 &  2 & 1 & 0 & 1 & 3 & 2 & 1 & 0 \\
%   \hline
%   Z        & 0      & 1      & 2      & 3      & 1 & 2 & 3 & 4 & 2 &  3 & 4 & 5 & 3 & 4 & 5 & 6 \\
%   \hline 
%   $P(W,Z)$ & 0.2392 & 0.1612 & 0.0832 & 0.0364 &   &   &   &   &   &    &   &   &   &   &   &  \\
%   \hline
%  \end{tabular}
% }
% \fi
%  \item Les v.a. $W$ et $Z$ sont-elles ind\'ependantes?
%   \ifcorrection
%   \textcolor{red}{
%    $P(Z|W) = \frac{P(Z, W)}{P(W)} =   $   %\frac{P(W)P(Z)}{{P(W)} = P(Z)$\\
%    Les 2 v.a. sont ind\'ependantes.
%   \fi
% \end{enumerate}
% 
% 
% %% probleme concret 2
% % (moyenne) 200 , 100 , 150, sigma = 50
% \exost On donne dans le tableau ci-dessous du nombre de ventes r\'ealis\'ees par une enseigne en fonction du jour dans le mois pour les 4 premiers mois de l'ann\'ee:\\
% \centering\begin{tabular}{|l|l|l|l|}
% \hline
% 		& du 1 au 10 & du 11 au 20 & du 21 au 31 \\
% \hline
% \hline 
% Janvier		& 291        & 78	   & 146 \\
% \hline 
% F\'evrier 	& 134        & 117         & 185 \\
% \hline 
% Mars	 	& 215        & 136         & 139 \\
% \hline
% Avril	 	& 184        & 133         & 143 \\
% \hline 
% \end{tabular}
% 
% \flushleft
% \begin{enumerate}
%  \item En notant $X$ la v.a. qui correspond au nombre de ventes et $Y$ la v.a. qui correspond au jour du mois, d\'efinissez la loi conjointe $p_{X,Y} = P(X=x,Y=y)$.
%  Vous utiliserez une pr\'ecision de 100 pour quantifier les valeurs de $X$.
%  \item Calculez les probabilit\'es suivantes: $P(X > 100)$, $P(X \leq 200)$, $P(100 < X \leq 200)$
%  \item Sachant qu'on est \`a la fin du mois, calculez $P(X > 100| Y > 21)$, $P(X \leq 300 | Y > 21)$
%  \item Sachant qu'on est au d\'ebut du mois, calculez $P(X > 100| Y \leq 10)$, $P(X \leq 300 | Y \leq 10)$
%  \item En utilisant le th\'eor\`eme de Bayes, calculez la probabilit\'e qu'on soit en d\'ebut du mois, sachant que le nombre de ventes est sup\'erieur \`a 200, 
%  en utilisant $P(X>200|Y \leq 10)$.
%  \item V\'erifiez le r\'esultat pr\'ec\'edent en utilisant la loi conjointe $p_{X,Y}$.
% \end{enumerate}


\end{document}

% End Of File

