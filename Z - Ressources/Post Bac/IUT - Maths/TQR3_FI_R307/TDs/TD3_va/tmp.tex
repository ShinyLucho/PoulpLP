\documentclass[a4paper]{article}
\usepackage[latin1]{inputenc}
\usepackage[T1]{fontenc}
\usepackage[francais]{babel}
\usepackage{entete}
\usepackage{noitemsep}
\usepackage{euscript} 
\usepackage{amsmath,amssymb,amsfonts,amsthm}
\usepackage{graphicx,graphics,epsfig,subfigure,color}
\usepackage{url}
\usepackage{algorithm2e}
\usepackage{multicol}
\usepackage{a4wide}
\usepackage{latexsym}
\usepackage{verbatim}
\setlength{\textheight}{23.5cm}
\setlength{\topmargin}{-1cm}
\setlength{\textwidth}{155mm}
\setlength{\oddsidemargin}{2mm}

%\renewcommand{\baselinestretch}{0.85}

%\input{macroAlgo}
\dontprintsemicolon

\setlength{\parindent}{0pt}  %%suppression indentation

%% ACRONYM DEFINITIONS

\newcommand{\lcx}[1]{\MakeLowercase{#1}}
%\renewcommand{\log}[0]{\text{ln}}
\renewcommand{\iint}[0]{\int\!\!\!\!\!\int}
\renewcommand{\iiint}[0]{\int\!\!\!\!\!\int\!\!\!\!\!\int}
\renewcommand{\iiiint}[0]{\int\!\!\!\!\!\int\!\!\!\!\!\int\!\!\!\!\!\int}
%% expression

%% Latin abrev.
\newcommand{\etc}[0]{\textit{etc} }
\newcommand{\etal}[0]{\textit{et al.} }
\newcommand{\eg}[0]{\textit{e.g.} }
\newcommand{\ie}[0]{\textit{i.e.} }
\newcommand{\cf}[0]{\textit{cf.} }

%% Mathematical coordinates
\newcommand{\meq}[0]{\!\!=\!\!}
\newcommand{\TW}[0]{(\tau, \Omega)}
\newcommand{\tw}[0]{(t, \omega)}
\newcommand{\tv}[0]{(t, \Omega)}
\newcommand{\ts}[0]{(t, s)}
\newcommand{\sw}[0]{(s, \omega)}
\newcommand{\zw}[0]{(z, \omega)}
\newcommand{\nm}[0]{[n,m]}
\newcommand{\uv}[0]{[u,v]}

%% Mathematical constants
\newcommand{\ww}[0]{|\omega|}   %% 1
\newcommand{\www}[0]{|\omega|}  %% []
\newcommand{\te}[0]{T_s}
\newcommand{\fs}[0]{F_s}
\newcommand{\ee}{\,\ensuremath{\mathbf{e}}}

%% Mathematical functions names
\renewcommand{\Re}[0]{\text{Re}}
\renewcommand{\Im}[0]{\text{Im}}
\newcommand{\ri}[0]{\text{Ri}}
\newcommand{\wvd}[0]{\text{WV}}
\newcommand{\cwt}[0]{\text{CW}}
\newcommand{\mwt}[0]{\text{MW}}
\newcommand{\sst}[0]{\text{SST}}
\newcommand{\swt}[0]{\text{SWT}}
\newcommand{\cre}[0]{\text{CRE}}
\newcommand{\card}[1]{\text{Card}(#1)}

%% Mathematical functions names
\newcommand{\rst}[0]{\text{RST}}
\newcommand{\vsst}[0]{\text{VSST}}
\newcommand{\osst}[0]{\text{OSST}}
\newcommand{\lmsst}[0]{\text{LMSST}}

%% Mathematical variables names
\newcommand{\TT}[0]{\mathcal{T}}
\newcommand{\DD}[0]{\mathcal{D}}
%% EOF

\newif\ifcorrection
%\correctiontrue   %% Final version
\correctionfalse   %% Reviewer's version

\newcommand{\universityname}{IUT d'\'Evry Val d'Essonne}
\newcommand{\deptname}{D\'epartement TC (S3)}
\newcommand{\years}{2023-2024}

\begin{document}
\selectlanguage{francais}
\author{D. Fourer, L. Lagon}
%------------------- TITRE -----------------------------------------
\date{Septembre 2023} 
\TDHead{\universityname}{\deptname}{R3.07 TQR3, \years}{\large TD3: Variables al\'eatoires}
%-------------------------------------------------------------------

\underline{Rappels:} 

\begin{itemize}
 \item $X:\Omega \rightarrow \mathbb{R}$: une v.a. \`a valeurs r\'eelles (affecte un nombre r\'eel \`a un \'ev\'enement de $\Omega$)
 \item $x$: une valeur possible prise par $X$
 \item $P(X=x)$: probabilit\'e que l'\'ev\'enement $\{X=x\}$ se r\'ealise
 \item $p_X(x)$: loi de probabilit\'e de $X$ d\'efinie comme $p_X(x)=P(X=x)$
 \item $F_X(x)=P(X\leq x)$: fonction de r\'epartition de $X$
\end{itemize}

~\par{}


\exost Une entreprise fabrique des interrupteurs avec
voyants lumineux. Un relev\'e statistique indique que 5\% des
interrupteurs fabriqu\'es sont d\'efectueux. Supposons que l'on
pr\'el\`eve successivement et au hasard de la production deux
interrupteurs. Notons $X$ la v.a. \og{}nombre d'interrupteurs
d\'efectueux dans l'\'echantillon pr\'elev\'e\fg{}. 

\begin{enumerate}
 \item D\'efinir la v.a. $X$.
 \ifcorrection
 \textcolor{red}{\\
 Avec $A=\{\text{l'interrupteur fonctionne}\}$, on a:
   $X: \{AA,A\bar{A},\bar{A}A,\bar{A}\bar{A} \} \rightarrow \{0,1,2\}$   
   }
 \fi
 \item D\'eterminer la loi de probabilit\'e de $X$.
 \ifcorrection
    \textcolor{red}{
   (Faire un arbre) avec $P(A)=0,95$.
   On a:
     \begin{tabular}{|l|l|l|l|l|}
      \hline
	$x$ 		& $P(X=x)$\\
      \hline
      0			& $0,9025$\\
      \hline
      1			& $0.0950$\\
      \hline
      2			& $0,0025$ \\
 \hline
 \end{tabular}
  }
  \fi
 \item D\'efinir la fonction de r\'epartition de $X$.
 \ifcorrection
    \textcolor{red}{\\
   On a:
     \begin{tabular}{|l|l|l|l|}
      \hline
	$x$ 		& 0		& 1		& 2 \\
      \hline
      $P(X=x)$		& $0,9025$	& $0.0950$	& $0,0025$\\
      \hline
      $F_X(x)$		& $0,9025$	& $0,9975$	& 1\\
      \hline
 \end{tabular}
  }
  \fi
 \item Quelle est la probabilit\'e pour qu'au plus, un interrupteur soit d\'efectueux?
 \ifcorrection
     \textcolor{red}{
     $P(X\leq 1) = F_X(1) = 0,9975$.
  }
  \fi
\end{enumerate}


\exost On tire successivement, avec remise, deux boules d'une urne qui contient une rouge R, une
verte V et une blanche B. 

\begin{enumerate}
 \item On gagne 1 euro avec R, 2 euros avec V et on perd 3 euros avec B. Quels sont les gains possibles? D\'efinir la v.a. $X$ qui repr\'esente le gain.
 \ifcorrection
 \textcolor{red}{
 On a $3^2=9$ tirages possibles:\\
 $\Omega = \{BB, BR, BV, RR, RV, VV\}$.\\
 La v.a. $X$ est d\'efinie comme suit:
 \begin{tabular}{|l|l|}
 \hline
  Ev\'enement 	& $x$ \\
 \hline
 BB		& -6 \\
 BR		& -2\\
 RB		& -2\\ 
 BV		& -1\\
 VB		& -1 \\ 
 RR		& 2\\
 RV		& 3\\
 VR		& 3 \\
 VV		& 4\\
 \hline
 \end{tabular}\\
Les gains possibles sont:
  \begin{tabular}{|l|l|l|}
 \hline
  $x$ 		& \# 	& $P(X=x)$\\
 \hline
 -6		& 1 	& $\frac{1}{9}$\\
 -2		& 2	& $\frac{2}{9}$\\
 -1		& 2	& $\frac{2}{9}$\\ 
 2		& 1	& $\frac{1}{9}$\\
 3		& 2	& $\frac{2}{9}$\\ 
 4		& 1	& $\frac{1}{9}$\\
 \hline
 \end{tabular}\\
$P(X=-6) = P(X=4) = P(X=2) = \frac{1}{9}$\\
$P(X=-1) = P(X=-2) = P(X=3) = \frac{2}{9}$\\
 $X:\Omega\rightarrow \{-6, -2, -1, 2, 3, 4 \}$\\
%L'esp\'erance vaut $E[X] = 
 }\fi
 \item Maintenant on change les r\`egles. On gagne 2 euros si les deux boules sont identiques et on perd 1 euro dans tous les autres cas.
 D\'efinir la v.a. $Y$ correspondante.
 \ifcorrection
  \textcolor{red}{
 On a $3^2=9$ tirages possibles:\\
 $\Omega = \{BB, BR,RB, BV,VB, RR, RV,VR, VV\}$.\\
 La v.a. est d\'efinie comme suit:
 \begin{tabular}{|l|l|}
 \hline
  \'Ev\'enement 	& $y$ \\
 \hline
 BB		& 2  \\
 RR		& 2 \\
 VV		& 2 \\
 BR		& -1 \\
 RB		& -1\\
 BV		& -1 \\
 VB		& -1\\
 RV		& -1 \\
 VR		& -1 \\
 \hline
 \end{tabular}\\
 $Y:\Omega\rightarrow \{-1 , 2 \}$\\
 On a $P(Y=2) = \frac{1}{3}$ et  $P(Y=-1) = \frac{2}{3}$\\
%L'esp\'erance vaut $E[X] = \frac{-1\times 2 + 2}{3} = 0$
 }\fi
 \item On revient au cas initial: on suppose \'equiprobable chaque tirage.
 \begin{enumerate}
  \item D\'efinir l'\'ev\'enement $A=\{\text{"obtenir un V sans B"}\}$.
  \ifcorrection
 \textcolor{red}{
  $A = \{VV,RV,VR\}$
  }
  \fi  
  \item Calculer la probabilit\'e de $A$.
  \ifcorrection
  \textcolor{red}{
  $P(A) = \frac{3}{9} = \frac{1}{3}$
  }
  \fi
  \item Quelle est l'image de $A$ par $X$?
  \ifcorrection
  \textcolor{red}{
 \begin{tabular}{|l|l|}
 \hline
  Ev\'enement 	& $x$ \\
 \hline
 RV		& 3\\
 VR		& 3 \\
 VV		& 4\\
 \hline
 \end{tabular}\\
  }
  \fi
  
  \item Quelle est la probabilit\'e de $X(A)$?
\ifcorrection
  \textcolor{red}{
  $P(X=3) = \frac{2}{9}$ et $P(X=4) = \frac{1}{9}$.\\
  On a aussi
  $P(X=3|A) = \frac{2}{3}$ et $P(X=4|A) = \frac{1}{3}$.
  }
  \fi
 \end{enumerate}
\item %On appelle $Z$ la v.a. qui \`a chaque \'ev\'enement fait correspondre la somme des valeurs attribu\'ees. 
D\'efinir la loi de probabilit\'e de $X$ (faire un tableau de $p_X$).
\ifcorrection
  \textcolor{red}{
\begin{tabular}{|l | l l l l l l|}
\hline
 $x:$ 	& -6		 & -2 		 & -1 		 & 2 		& 3		 & 4 \\
 \hline
 $p_X:$	& $\frac{1}{9}$  & $\frac{2}{9}$ & $\frac{2}{9}$ & $\frac{1}{9}$&$\frac{2}{9}$	 & $\frac{1}{9}$\\
 \hline
\end{tabular}
}
  \fi
\item Quelle est la probabilit\'e d'obtenir au moins $0$? 
\ifcorrection
  \textcolor{red}{%~\\
$P(X\geq 0) = \sum_{x \geq 0}p_X(x) = P(X=2) + P(X=3) + P(X=4) = \frac{4}{9}$
}
\fi
\item Quelle est la probabilit\'e de gagner de l'argent \`a ce jeu?
\ifcorrection
  \textcolor{red}{cf. r\'eponse ci-dessus, $P(X>0)=P(X\geq 0)$}
\fi
\item Que peut on esp\'erer gagner? (calculer l'esp\'erance de $X$ d\'efinie par $E[X]=\sum_x x P(X=x)$)
\ifcorrection
  \textcolor{red}{
 \begin{align}
 E[X] &= -6 \times \frac{1}{9} + -2 \times \frac{2}{9} + (-1) \times \frac{2}{9} + 2 \times \frac{2}{9} + 3 \times \frac{2}{9} + 4 \times \frac{1}{9}\\
      &= 0
\end{align}}
\fi
\end{enumerate}
%

\exost Les 3 commerciaux d'une entreprise ont respectivement une probabilit\'e de 0,1 - 0,2 - 0,3
de conclure un contrat chaque jour ouvrable. La probabilit\'e pour chacun d'eux de signer plusieurs contrats
est nulle.
\begin{enumerate}
 \item D\'efinir $X$, la v.a. du nombre de contrats sign\'es pour l'entreprise un jour donn\'e.
 \ifcorrection
  \textcolor{red}{
  Faire un arbre, chaque commercial signe ou ne signe pas de contrat avec sa probabilit\'e. $2^3=8$ feuilles.\\
  $X: \{\bar{S1}\bar{S2}\bar{S3},S1,S2,S3,S1S2,S1S3,S2S3, S1S2S3\} \rightarrow \{0, 1, 2, 3\}$
  %
  }
\fi
 \item D\'efinir la loi de probabilit\'e de $X$.
 \ifcorrection
  \textcolor{red}{
  \begin{tabular}{|l|l|l|}
 \hline
  \'Ev\'enement 		& $x$ 	& $p(x=X)$\\
 \hline
 $\bar{S1}\bar{S2}\bar{S3}$	& 0  	& $(1-0,1)(1-0,2)(1-0,3) \approx 0,504$\\
 \hline
 $S1\bar{S2}\bar{S3}$		& 1 	& $0,1 (1-0,2) (1-0,3)\approx 0,056$\\
 $S2\bar{S1}\bar{S3}$		& 1	& $0,2 (1-0,1) (1-0,3) \approx 0,126$ \\
 $S3 \bar{S1}\bar{S2}$		& 1	& $\approx 0,216$ \\
 \hline
 $S1S2\bar{S3}$			& 2	& $\approx 0,014$\\
 $S1S3\bar{S2}$			& 2 	& $\approx 0,024$\\
 $S2S3\bar{S1}$			& 2	& $\approx 0,054$\\
 \hline
 $S1S2S3$			& 3 	& $\approx 0,006$\\
 \hline
 \end{tabular}\\
 }
 \fi
 \item D\'efinir sa fonction de r\'epartition.
  \ifcorrection
  \textcolor{red}{
  On trie par ordre croissant les valeurs $x$ et on fait la somme des probabilit\'es associ\'ees \`a chaque valeur:
  \begin{tabular}{|l|l|l|l|l|}
 \hline
 $x$ 			& 0 	& 1 	& 2 		& 3\\
 \hline
 $P_X(x)$		& 0,504	& 0.398	& 0.092		& 0,006\\
 \hline
 $F_X(x)$		& 0,504	& 0,902	& 0.994		& 1\\
 \hline
 \end{tabular}}
 \fi
 \item Quelle est la probabilit\'e de signer 0, 1, 2, ou 3 contrat(s)?
   \ifcorrection
  \textcolor{red}{
  On utilise $P(X=x)$ calcul\'ee plus haut.
  }
  \fi
 \item Quelle est la probabilit\'e de signer au moins un contrat?
    \ifcorrection
  \textcolor{red}{
  $P(X\geq 1) = \sum_{x>0}P(X=x) = 0,496$
  }
  \fi
\end{enumerate}

\end{document}

% End Of File

