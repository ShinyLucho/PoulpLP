\documentclass[a4paper]{article}
\usepackage[latin1]{inputenc}
\usepackage[T1]{fontenc}
\usepackage[francais]{babel}
\usepackage{entete}
\usepackage{noitemsep}
\usepackage{euscript} 
\usepackage{amsmath,amssymb,amsfonts,amsthm}
\usepackage{graphicx,graphics,epsfig,subfigure,color}
\usepackage{url}
\usepackage{algorithm2e}
\usepackage{multicol}
\usepackage{a4wide}
\usepackage{latexsym}
\usepackage{verbatim}
\setlength{\textheight}{23.5cm}
\setlength{\topmargin}{-1cm}
\setlength{\textwidth}{155mm}
\setlength{\oddsidemargin}{2mm}

%\renewcommand{\baselinestretch}{0.85}

%\input{macroAlgo}
\dontprintsemicolon

\setlength{\parindent}{0pt}  %%suppression indentation

%% ACRONYM DEFINITIONS

\newcommand{\lcx}[1]{\MakeLowercase{#1}}
%\renewcommand{\log}[0]{\text{ln}}
\renewcommand{\iint}[0]{\int\!\!\!\!\!\int}
\renewcommand{\iiint}[0]{\int\!\!\!\!\!\int\!\!\!\!\!\int}
\renewcommand{\iiiint}[0]{\int\!\!\!\!\!\int\!\!\!\!\!\int\!\!\!\!\!\int}
%% expression

%% Latin abrev.
\newcommand{\etc}[0]{\textit{etc} }
\newcommand{\etal}[0]{\textit{et al.} }
\newcommand{\eg}[0]{\textit{e.g.} }
\newcommand{\ie}[0]{\textit{i.e.} }
\newcommand{\cf}[0]{\textit{cf.} }

%% Mathematical coordinates
\newcommand{\meq}[0]{\!\!=\!\!}
\newcommand{\TW}[0]{(\tau, \Omega)}
\newcommand{\tw}[0]{(t, \omega)}
\newcommand{\tv}[0]{(t, \Omega)}
\newcommand{\ts}[0]{(t, s)}
\newcommand{\sw}[0]{(s, \omega)}
\newcommand{\zw}[0]{(z, \omega)}
\newcommand{\nm}[0]{[n,m]}
\newcommand{\uv}[0]{[u,v]}

%% Mathematical constants
\newcommand{\ww}[0]{|\omega|}   %% 1
\newcommand{\www}[0]{|\omega|}  %% []
\newcommand{\te}[0]{T_s}
\newcommand{\fs}[0]{F_s}
\newcommand{\ee}{\,\ensuremath{\mathbf{e}}}

%% Mathematical functions names
\renewcommand{\Re}[0]{\text{Re}}
\renewcommand{\Im}[0]{\text{Im}}
\newcommand{\ri}[0]{\text{Ri}}
\newcommand{\wvd}[0]{\text{WV}}
\newcommand{\cwt}[0]{\text{CW}}
\newcommand{\mwt}[0]{\text{MW}}
\newcommand{\sst}[0]{\text{SST}}
\newcommand{\swt}[0]{\text{SWT}}
\newcommand{\cre}[0]{\text{CRE}}
\newcommand{\card}[1]{\text{Card}(#1)}

%% Mathematical functions names
\newcommand{\rst}[0]{\text{RST}}
\newcommand{\vsst}[0]{\text{VSST}}
\newcommand{\osst}[0]{\text{OSST}}
\newcommand{\lmsst}[0]{\text{LMSST}}

%% Mathematical variables names
\newcommand{\TT}[0]{\mathcal{T}}
\newcommand{\DD}[0]{\mathcal{D}}
%% EOF

\newif\ifcorrection
%\correctiontrue   %% Final version
\correctionfalse   %% Reviewer's version

\newcommand{\universityname}{IUT d'\'Evry Val d'Essonne}
\newcommand{\deptname}{D\'epartement TC (S3)}
\newcommand{\years}{2023-2024}

\begin{document}
\selectlanguage{francais}
\author{D. Fourer, L. Lagon}
%------------------- TITRE -----------------------------------------
\date{Septembre 2022} 
\TDHead{\universityname}{\deptname}{R3.07 TQR3, \years}{\large TD2: Probabilit\'es conditionnelles}
%-------------------------------------------------------------------

\underline{Rappels:} 

\begin{itemize}
 \item probabilit\'e que $A$ et $B$ se r\'ealisent: $P(A \cap B)$
 \item $A$ et $B$ sont ind\'ependants si $P(A \cap B) = P(A) P(B)$
 \item $A$ et $B$ sont incompatibles si $P(A \cap B) = 0$
 \item probabilit\'e de $A$ sachant que $B$ s'est r\'ealis\'e:  $P(A|B) = \frac{P(A \cap B)}{P(B)}$
 \item th\'eor\`eme de Bayes: $P(A|B) = \frac{P(A)P(B|A)}{P(B)} = \frac{P(A)P(B|A)}{P(A)P(B|A) + P(A^c)P(B|A^c)}$
\end{itemize}

~\par{}


%% ex: proba totale (cours)
\exost On dispose de 3 urnes $U_1$, $U_2$, $U_3$ contenant chacune $10$ boules de couleurs.
\begin{itemize}
 \item $U_1$ contient 3 boules blanches
 \item $U_2$ contient 5 boules blanches
 \item $U_3$ contient 2 boules blanches
\end{itemize}

Si on tire une boule au hasard dans une des 3 urnes (tirage \'equiprobable).
\begin{enumerate}
 \item Quelle est la probabilit\'e d'obtenir une boule blanche? Une boule d'une autre couleur?
 \ifcorrection
 \textcolor{red}{On note $B$ l'\'ev\'enement ``on tire une boule blanche'' et $A_i$ l'\'ev\'enement
 on tire dans l'urne $U_i$.\\
 $\{A_1,A_2,A_3 \}$ forme une partition de $\Omega$. Le th\'eor\`eme de probabilit\'e totale
 indique que:
 \begin{align}
\nonumber  P(B) &= P(A_1)P(B|A_1) + P(A_2)P(B|A_2) + P(A_3)P(B|A_3)\\
\nonumber      	&= \frac{1}{3}\frac{3}{10} + \frac{1}{3}\frac{5}{10} + \frac{1}{3}\frac{2}{10}\\
		&= \frac{1}{3}
 \end{align}
 }
 \fi
 \item M\^eme question si on a $10\%$ de chance de tirer dans l'urne $U_2$ et $60\%$ de chance de choisir 
 l'urne $U_3$.
 \ifcorrection
 \textcolor{red}{
 On a $P(A_1) = \frac{3}{10}$, $P(A_2) = \frac{1}{10}$ et $P(A_3) = \frac{6}{10}$
 \begin{align}
\nonumber  P(B) &= P(A_1)P(B|A_1) + P(A_2)P(B|A_2) + P(A_3)P(B|A_3)\\
\nonumber      	&= \frac{3}{10}\frac{3}{10} + \frac{1}{10}\frac{5}{10} + \frac{6}{10}\frac{2}{10}\\
		&= \frac{9}{100} + \frac{5}{100} + \frac{12}{100}\\
		&= \frac{26}{100} = \frac{13}{50}
 \end{align}}
 \fi
 \item M\^eme question si on r\'ep\`ete l'exp\'erience 3 fois avec remise (tirage \'equiprobable).
 \ifcorrection
  \textcolor{red}{
  (faire un arbre pour repr\'esenter les possibilit\'es)
 On consid\`ere l'\'ev\'enement $N = B^c = \{\text{``on tire 3 boules noires''}\}$.
 $P(N) = (1-P(B))^3 = \left(\frac{2}{3}\right)^3 = \frac{8}{27} $.
 Donc apr\`es 3 tirages, $P(B) = 1-P(N) = \frac{19}{27} \approx 0.70$.
 }
 \fi
\end{enumerate}

\exost Trois responsables Alexandre, Bruno et C\'edric, peuvent signer des contrats pour leur entreprise.
Ils \'etudient respectivement 40\%, 30\% et 20\% des contrats de l'entreprise.
Le taux de r\'eussite (rapport entre le nombre de contrats sign\'es et celui des contrats \'etudi\'es)
de chaque responsable est respectivement de 50\%, 70\% et 80\%.
\begin{enumerate}
 \item Quel est le nombre de contrats sign\'es par Alexandre si l'entreprise \'etudie 1000 contrats?
 \ifcorrection
 \textcolor{red}{
 Alexandre signe 50\% des 40\% \'etudi\'es soit $0.5 \times 0.4 \times 1000 = 200$ contrats.
 }
\fi
 \item Quel est le pourcentage de signatures de contrats r\'eussies par rapport au nombre de contrats total \'etudi\'e par l'entreprise?
 \ifcorrection
 \textcolor{red}{
 L'entreprise a sign\'e $1000 \times (0.5\times 0.4 + 0.7\times 0.3 + 0.8\times 0.2 ) = 1000 \times (0.20 + 0.21 + 0.16) = 570 $ contrat.
 L'entreprise signe $57\%$ de ses contrats.
 }
\fi
 \item Sachant qu'un contrat est sign\'e, quelle est la probabilit\'e que ce soit par Alexandre.
 \ifcorrection
 \textcolor{red}{
 On note $A$ l'\'ev\'enement $\{ \text{Alexandre \'etudie un contrat} \}$, et $S$ l'\'ev\'enement un contrat est sign\'e.
 Nous avons $P(A) = 0.4$, $P(S) = 0.57$ et $P(S|A) = 0.5$
 Donc $P(A|S) = \frac{P(A \cap S)}{P(S)} = \frac{P(A) P(S|A)}{P(S)} = \frac{0.4 \times 0.5}{0.57} \approx 0.35$
 }
\fi
\end{enumerate}

%% Maladie Bayes
\exost Vous \^etes directeur de cabinet du ministre de la sant\'e. Une maladie est pr\'esente dans la population 
touche une personne sur 10000. Un responsable d'un grand laboratoire pharmaceutique vient vous vanter son nouveau test de d\'epistage: 
\begin{itemize}
 \item si une personne est malade, le test est positif \`a 99\%.
 \item si une personne n'est pas malade, le test est positif \`a 0.01\%. 
\end{itemize}
Toutefois, avant d'autoriser la commercialisation de ce test, vous faites appel au statisticien du minist\`ere:
ce qui vous int\'eresse, ce n'est pas vraiment les r\'esultats pr\'esent\'es par le laboratoire mais la probabilit\'e 
qu'une personne soit malade si le test est positif.
\begin{enumerate}
 \item En utilisant le th\'eor\`eme de Bayes, calculez cette probabilit\'e.
 \item Qu'en conclure? Pouvez-vous dire que le test propos\'e est fiable?
\end{enumerate}
 \ifcorrection
 \textcolor{red}{
  On note $M$ l'\'ev\'enement $\{\text{\^etre malade}\}$ et $T$ l'\'ev\'enement $\{\text{le test est positif}\}$.
  Nous avons $P(M) = \frac{1}{10000}=10^{-4}$, $P(T|M) = 0.99$ et $P(T|M^c)=0.0001=10^{-4}$ et donc $P(M^c)=1-\frac{1}{10000}$.
  En appliquant le th\'eor\`eme de Bayes nous avons:
  \begin{align}
   P(M|T) = \frac{P(M) P(T|M)}{P(T)} = \frac{P(M) P(T|M)}{P(M)P(T|M) + P(M^c)P(T|M^c)} = \frac{10^{-4} \times 0.99}{10^{-4} \times 0.99 + (1-10^{-4}) \times 10^{-4} } \approx 0.49. 
  \end{align}
  Il y a moins d'une chance sur deux que le patient soit malade si le test est positif. On ne peut donc pas vraiment dire que le test est fiable.
 }
 \fi

 \exost Un tireur a probabilit\'e de $\frac{2}{3}$ d'atteindre sa cible. Combien de balles faut il lui donner
pour qu'il y ait au moins 99 chances sur 100 que la cible soit atteinte?

 \ifcorrection
  \textcolor{red}{
  On note $A$ l'\'ev\'enement $\{\text{Le tireur a atteint sa cible} \}$. Alors $P(A^c) = \frac{1}{3}$.
  Comme chaque tir est ind\'ependant, alors la probabilit\'e de ne jamais atteindre sa cible au bout de $n$
  tirs est de $\frac{1}{3^n}$. On doit r\'esoudre l'\'equation:
  \begin{align}
   1 - \frac{1}{3^n} &> \frac{99}{100}\\
   1 - 0.99 &> \frac{1}{3^n}\\
   3^n &> \frac{1}{0.01}\\
   n \cdot \log(3) &> \log(100) \\
   n &> 4.19
  \end{align}
  On en d\'eduit qu'il faut lui donner au moins 5 balles.
 }
 \fi




%% Paradoxe des anniversaires












% 
% 
% \exost
% On consid\`ere que l'on dispose de l'alphabet latin (sans accent) de 26 lettres.
% \begin{enumerate}
%  \item Combien de mots diff\'erents peut on \'ecrire en utilisant toutes les lettres du mot \og{}facile\fg{}?
%    \ifcorrection
%  \textcolor{red}{
%  ``Facile'' s'\'ecrit avec 6 lettres, donc il existe $P_6=6!=720$ permutations possibles.
%  }
%  \fi
%  \item Combien de mots diff\'erents de exactement 6 lettres peut on \'ecrire avec tout l'alphabet? (on autorise les r\'ep\'etitions)
%    \ifcorrection
%  \textcolor{red}{
%  On consid\`ere un tirage ordonn\'e de 6 \'el\'ement avec remise, soit $26^6 = 308~915~776$ mots possibles.
%  }
%  \fi
%  \item Combien de mots diff\'erents de exactement 6 lettres peut on \'ecrire sans r\'ep\'etition?
%    \ifcorrection
%  \textcolor{red}{
%  On consid\`ere un tirage ordonn\'e de 6 \'el\'ement sans remise, soit $A_{26}^6 = \frac{26!}{(26-6)!} = 165~765~600$ mots possibles.
%  }
%  \fi
%  \item Combien de mots diff\'erents de 5 lettres ou moins peut on \'ecrire sans r\'ep\'etition (dans le m\^eme mot)?
%    \ifcorrection
%  \textcolor{red}{
%  On consid\`ere la somme de tous les tirage ordonn\'es de 5 \'el\'ements ou moins sans remise, soit $A_{26}^1 + A_{26}^2 + A_{26}^3 + A_{26}^4 + A_{26}^5 = 26 + 26\times 25 + 26\times 25\times 24 + 26\times 25\times 24\times 23 + A_{26}^5 = 8~268~676$ mots possibles.
%  }
%  \fi
%  \item On tire d\'esormais au hasard 3 lettres dans l'ordre et sans r\'ep\'etition. Quelle est la probabilit\'e d'obtenir le mot \og{}IUT\fg{} si tous les tirages sont \'equiprobables?.
%   \ifcorrection
%   \textcolor{red}{
%    On consid\`ere le nombre d'arrangements de 3 lettres parmi 26 lettres: $\card{\Omega}=A_{26}^3=\frac{26!}{(26-3)!} = 15~600$.\\
%    Si on note $A$ l'\'ev\'enement $\{\text{``obtenir le mot IUT''} \}$, alors $P(A) = \frac{1}{15~600}$
%   }
%   \fi
% \end{enumerate}
% 
% 
% 
% \exost 
% On propose \`a un examen un questionnaire \`a choix multiples (QCM) avec 8 questions.
% Pour chaque question, il y a 3 r\'eponses possibles dont une seule est correcte.
% Le candidat d\'ecide de r\'epondre au hasard en ne cochant qu'une seule case \`a chaque question.
% \begin{enumerate}
%  \item Combien il y a-t-il de fa\c{c}ons diff\'erentes de remplir le questionnaire?
%   \ifcorrection
%  \textcolor{red}{
%   On applique la formule d'un choix de 5 \'el\'ements ordonn\'es parmis 3 avec remise:
%   Il y a donc $\card{\Omega} = 3^8= 6561$ choix possibles.
%  }
%  \fi
%  \item Combien de grilles diff\'erentes ne comportent qu'une seule r\'eponse fausse.
%   \ifcorrection
%  \textcolor{red}{
%  Pour chaque question il y a 2 r\'eponses fausses. En supposant, toutes les autres r\'eponses justes, on fait la somme des r\'eponses fausses
%  contenues dans le QCM, soit $2\times 8 = 16$ grilles diff\'erentes avec une seule faute.
%  }
%  \else
%  \fi
%  \item Combien de grilles diff\'erentes possibles sont enti\`erement fausses?
%  \ifcorrection
%  \textcolor{red}{
%  Pour chaque question il y a 2 r\'eponses fausses. En supposant, toutes les autres r\'eponses justes, on fait la somme des r\'eponses fausses
%  soit $2^8 = 256$ grilles diff\'erentes enti\`rement fausses.
%  }
%   \else
%  \fi
%   \item Combien de grilles diff\'erentes possibles ont au moins une bonne r\'eponse?
%   \ifcorrection
%  \textcolor{red}{
%  On note $A$ l'\'ev\'enement $\{\text{``La grille est enti\`erement fausse''}\}$, donc $A^c = \{ \text{``La grille contient une bonne r\'eponse''}\}$.
%  Comme on sait que $\card{A}=256$ (cf. r\'eponse pr\'ec\'edente), alors $\card{A^c} = 6561-256 = 6305$.
%  }
%  \else
%  \fi
% \end{enumerate}
% 
% 
% 
% %% Denombrement
% \exost
% Les membres d'une association de 20 personnes (13 hommes et 7 femmes) souhaitent constituer un bureau de 3 personnes (un(e) pr\'esident(e), un(e) tr\'esorier(e)
% et un(e) secr\'etaire).
% \begin{enumerate}
%  \item Combien de bureaux (groupe de 3 personnes) diff\'erents peuvent \^etre constitu\'es \`a partir de ces 20 personnes?
%  \ifcorrection
%  \textcolor{red}{
%   On applique la formule d'un tirage ordonn\'e de 3 \'el\'ements parmis 20 $A_{20}^3 = \frac{20!}{17!} = 20 \times 19 \times 18 = 6840$.
%   }
%   \else
%  \fi
%  \item Combien de bureaux diff\'erents ayant une femme pr\'esidente peuvent \^etre constitu\'es?
%   \ifcorrection
%   \textcolor{red}{
%   On applique la formule d'un tirage ordonn\'e de 2 \'el\'ements parmis 19: $A_{19}^2 = \frac{19!}{17!} = 19 \times 18  = 342$.
%   Comme il y a 7 femmes, alors au total il y a $7 \times 342 = 2394$ bureaux possibles avec 1 femme pr\'esidente.
%   (et donc $13 \times 342 = 4446$ bureaux avec 1 homme pr\'esident). On v\'erifie que $2394 + 4446 = 6840$ correspond au total.  
%   }
%  \fi
%  \item En supposant \'equiprobable le choix de chaque candidat. Quelle est la probabilit\'e pour que le bureau soit compos\'e d'au moins une femme?
%   \ifcorrection
%   \textcolor{red}{
%   %On note $\Omega$ l'ensemble des bureaux possibles tel que $\card{\Omega} = C_{20}^3 = 1140$.
%   -On note $F$ l'\'ev\'enement, $\{\text{``une femme est dans le bureau''}\}$ et son compl\'ementaire $F^c=\{\text{``Il n'y a pas de femme dans le bureau''} \}$.
%   Le tirage de chaque candidat est ind\'ependant donc:
%   $P(F^c) = \frac{13}{20} \frac{12}{19} \frac{11}{18} = \frac{1716}{6840} \approx 0,25$.
%   Donc $P(F) = 1 - P(F^c) \approx 0,75$.\\
%   -Autre solution: Il y a $A_{13}^3$ arrangements de 3 hommes, donc $P(F^c) = \frac{A_{13}^3}{A_{20}^3} = \frac{\frac{13!}{(13-3)!}}{\frac{20!}{(20-3)!}}=\frac{13\times 12 \times 11}{20\times 19 \times 18}\approx 0,25$\\
%   -Autre solution: On \'enum\`ere toutes les combinaisons (avec au moins 1F): 3F: $C_7^3$, 2F1M: $C_7^2 C_{13}^1$, 1F2M: $C_7^1C_{13}^2$, donc
%   $P(F) = \frac{C_7^3 + (C_7^2C_{13}^1) + (C_7^1C_{13}^2)}{C_{20}^3} = \frac{854}{1140} \approx 0,75$.
%   }
%   \else
%  \fi
% \end{enumerate}
% 
% \exost 
% Un sac contient 13 boules noires et 2 boules rouges. Combient faut il en tirer simultan\'ement
% pour que la probabilit\'e d'obtenir au minimum une boule rouge soit sup\'erieure \`a $\frac{6}{7}$?
% 
% \ifcorrection
%   \textcolor{red}{
%   On note $x$ l'inconnue (le nombre de boules tir\'ees) et $A^c$ l'\'ev\'enement compl\'ementaire $\{\text{``on ne tire aucune boule rouge''}\}$.
%   On a $\card{A^c} = C_{13}^x$ et $\card{\Omega} = C_{15}^x$.\\
%   Il faut alors r\'esoudre l'in\'equation:
%   $P(A) = 1 - \frac{C_{13}^x}{C_{15}^x} > \frac{6}{7}$.
%   On a donc:
%   \begin{align}
%    \frac{1}{7} 			&> \frac{\frac{13!}{x!(13-x)!}}{\frac{15!}{x!(15-x)!}} = \frac{13!(15-x)!}{15!(13-x)!} = \frac{(15-x)(14-x)}{15\times 14}\\
%    \frac{15 \times 14}{7}	&> x^2 - 29x + 210\\
%    0 				&> x^2 - 29x + 180
%   \end{align}
%   On pose $\Delta = b^2 - 4 ac$ ($b=-29$, $a=1$ et $c=180$) donc $\Delta = 121 = 11^2$.\\
%   On a donc (la plus petite solution) $x>\frac{-b-\sqrt{\Delta}}{2a} = \frac{29-11}{2} = 9$.\\
%   L'autre solution $x_2>\frac{-b+\sqrt{\Delta}}{2a} = 20$ est bien plus grande que 9 mais sans int\'er\^et car on cherche le nombre
%   minimum de tirage.
%   }
%   \else
%  \fi

\end{document}

% End Of File

