\documentclass[a4paper, 11pt]{article}
\usepackage[utf8]{inputenc}
%\usepackage[latin1]{inputenc}
\usepackage[T1]{fontenc}
\usepackage[francais]{babel}
\usepackage{entete}
\usepackage{noitemsep}
\usepackage{euscript} 
\usepackage{amsmath,amssymb,amsfonts,amsthm}
\usepackage{graphicx,graphics,epsfig,subfigure,color}
\usepackage{url}
\usepackage{algorithm2e}
\usepackage{multicol}
\usepackage{a4wide}
\usepackage{latexsym}
\usepackage{verbatim}
\setlength{\textheight}{23.8cm}
\setlength{\topmargin}{-2.5cm}
\setlength{\textwidth}{165mm}
\setlength{\oddsidemargin}{0mm}
\usepackage{eurosym}

%\renewcommand{\baselinestretch}{0.85}

%\input{macroAlgo}
\dontprintsemicolon

\setlength{\parindent}{0pt}  %%suppression indentation

%% ACRONYM DEFINITIONS

\newcommand{\lcx}[1]{\MakeLowercase{#1}}
%\renewcommand{\log}[0]{\text{ln}}
\renewcommand{\iint}[0]{\int\!\!\!\!\!\int}
\renewcommand{\iiint}[0]{\int\!\!\!\!\!\int\!\!\!\!\!\int}
\renewcommand{\iiiint}[0]{\int\!\!\!\!\!\int\!\!\!\!\!\int\!\!\!\!\!\int}
%% expression

%% Latin abrev.
\newcommand{\etc}[0]{\textit{etc} }
\newcommand{\etal}[0]{\textit{et al.} }
\newcommand{\eg}[0]{\textit{e.g.} }
\newcommand{\ie}[0]{\textit{i.e.} }
\newcommand{\cf}[0]{\textit{cf.} }

%% Mathematical coordinates
\newcommand{\meq}[0]{\!\!=\!\!}
\newcommand{\TW}[0]{(\tau, \Omega)}
\newcommand{\tw}[0]{(t, \omega)}
\newcommand{\tv}[0]{(t, \Omega)}
\newcommand{\ts}[0]{(t, s)}
\newcommand{\sw}[0]{(s, \omega)}
\newcommand{\zw}[0]{(z, \omega)}
\newcommand{\nm}[0]{[n,m]}
\newcommand{\uv}[0]{[u,v]}

%% Mathematical constants
\newcommand{\ww}[0]{|\omega|}   %% 1
\newcommand{\www}[0]{|\omega|}  %% []
\newcommand{\te}[0]{T_s}
\newcommand{\fs}[0]{F_s}
\newcommand{\ee}{\,\ensuremath{\mathbf{e}}}

%% Mathematical functions names
\renewcommand{\Re}[0]{\text{Re}}
\renewcommand{\Im}[0]{\text{Im}}
\newcommand{\ri}[0]{\text{Ri}}
\newcommand{\wvd}[0]{\text{WV}}
\newcommand{\cwt}[0]{\text{CW}}
\newcommand{\mwt}[0]{\text{MW}}
\newcommand{\sst}[0]{\text{SST}}
\newcommand{\swt}[0]{\text{SWT}}
\newcommand{\cre}[0]{\text{CRE}}
\newcommand{\card}[1]{\text{Card}(#1)}

%% Mathematical functions names
\newcommand{\rst}[0]{\text{RST}}
\newcommand{\vsst}[0]{\text{VSST}}
\newcommand{\osst}[0]{\text{OSST}}
\newcommand{\lmsst}[0]{\text{LMSST}}

%% Mathematical variables names
\newcommand{\TT}[0]{\mathcal{T}}
\newcommand{\DD}[0]{\mathcal{D}}
%% EOF

\newif\ifcorrection
%\correctiontrue   %% avec correction
\correctionfalse   %% sans correction

\newcommand{\universityname}{IUT d'\'Evry Val d'Essonne}
\newcommand{\deptname}{D\'epartement TC (S1)}
\newcommand{\years}{2023-2024}

\begin{document}
\selectlanguage{francais}
\author{D. Fourer, L. Lagon}
%------------------- TITRE -----------------------------------------
\date{Septembre 2022} 
\TDHead{\universityname}{\deptname}{R1.07, \years}{\large TD3: Math\'ematiques financi\`eres et statistique descriptive}
%-------------------------------------------------------------------


\exost En 2000, Jean a reçu $80$\,\euro{} d'\'etrennes et depuis, ses étrennes augmentent chaque ann\'ee de $6$\,\euro. 
Pour tout entier $n$, on note $u_n$ le montant des \'etrennes re\c{c}ues par Anne l'ann\'ee $2000+ n$. On a donc $u_0= 80$.
  \begin{enumerate}
    \item Calculer les \'etrennes qu'a reçues Anne en 2001, puis en 2002.
    \item Donner la nature de la suite $(u_n)$. (On justifiera soigneusement et l'on pr\'ecisera les param\`etres.)
    \item En déduire l'expression de $u_n$ en fonction de $n$.
    \item On note $S_n$ la somme des \'etrennes re\c{c}ues par Anne de l'ann\'ee 2000 jusqu'à l'année $2000+ n$. On a donc $S_n= u_0+ u_1+ \cdots+ u_n$. Calculer $S_{15}$.
  \end{enumerate}
%   \emph{Formulaire.}~La somme $S$ des $n+ 1$ premiers termes d'une suite arithmétique $(u_n)$ est donnée par:
%   \[
%     S= u_0+ u_1+ \cdots+ u_n= (n+ 1)\frac{u_0+ u_n}{2}.
%   \]
% \end{exercise}


\exost \`A la naissance de leur fils en 2011, des parents bloquent une somme d'argent afin de pouvoir financer d'\'eventuelles \'etudes \`a sa majorit\'e.
La banque leur propose un placement à int\'er\^ets simples à $5$\,\% par an. Ils d\'ecident de simuler un placement de $5\,000$\,\euro. On note $u_n$ la somme disponible à l'ann\'ee $2011+ n$.
  \begin{enumerate}
    \item Donner $u_1$, $u_2$, $u_3$ et $u_4$.
    \item Exprimer $u_{n+ 1}$ en fonction de $u_n$. Quelle est la nature de la suite $(u_n)$? (On préciser les paramètres.)
    \item Calculer le taux d'évolution exprimé en pourcentage, arrondi au centième, du capital sur toute la durée du placement.
    \item Finalement les parents déposent $10\,000$\,\geneuro, au lieu de $5\,000$. Quelle sera la somme disponible à la majorité de leur fils?
  \end{enumerate}

\exost Le g\'erant d'une petite boutique a relev\'e le nombre d'articles vendus par jour. Son relev\'e a porté sur les ventes des mois de Mars et Avril, ce qui correspond \`a $52$~jours de vente.
Le relev\'e des observations quotidiennes est le suivant:
  \begin{center}
    \begin{tabular}{lllllllllllll}
       7 & 13 & 8  & 10 & 9  & 12 & 10 & 8  & 9  & 10 & 6  & 14 & 7  \\
      15 & 9  & 11 & 12 & 11 & 12 & 5  & 14 & 11 & 8  & 10 & 14 & 12 \\
       5 & 7  & 13 & 12 & 16 & 11 & 9  & 11 & 11 & 12 & 12 & 15 & 14 \\
       5 & 14 & 9  & 9  & 14 & 13 & 11 & 10 & 11 & 12 & 9  & 15 & 8
    \end{tabular}
  \end{center}
  \begin{enumerate}
    \item Quel est le caract\`ere \'etudi\'e? Pr\'eciser son type.
    \item D\'eterminer les effectifs, fr\'equences, ECC, FCC, ECD et FCD.
    \item Calculez l'esp\'erance, la variance ainsi que l'\'ecart-type de cette variable statistique.
  \end{enumerate}
  
  \clearpage
\exost   Une entreprise de transport rel\`eve les temps de chaque pause des ses chauffeurs sur une semaine et compile ces données dans le tableau suivant:
  \begin{center}
    \begin{tabular}{|l|l|l|l|l|l|l|}
      \hline
      Durée de pause (minutes)  & [0; 10[ & [10; 30[ & [30; 50[ & [50; 60[ & [60; 90[ & [90; 120] \\
      \hline
      Effectif cumulé croissant &   105   &   508    &   709    &   877    &   903    &    912    \\
      \hline
    \end{tabular}
  \end{center}
  \begin{enumerate}
    \item D\'eterminer le caract\`ere \'etudi\'e. Pr\'eciser son type.
    \item Combien de pauses de moins d'une heure ont-elles \'et\'e prises?
    \item Calculer le nombre de pauses durant entre $60$ et $90$~minutes qui ont \'et\'e prises.
    \item D\'eterminer le nombre de pauses de plus de $30$~minutes qui ont \'et\'e prises.
    \item Estimez la dur\'ee moyenne des pauses (esp\'erance)
  \end{enumerate}
 

\end{document}

% End Of File

