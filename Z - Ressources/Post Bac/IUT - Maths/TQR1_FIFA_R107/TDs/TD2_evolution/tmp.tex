\documentclass[a4paper, 11pt]{article}
\usepackage[utf8]{inputenc}
%\usepackage[latin1]{inputenc}
\usepackage[T1]{fontenc}
\usepackage[francais]{babel}
\usepackage{entete}
\usepackage{noitemsep}
\usepackage{euscript} 
\usepackage{amsmath,amssymb,amsfonts,amsthm}
\usepackage{graphicx,graphics,epsfig,subfigure,color}
\usepackage{url}
\usepackage{algorithm2e}
\usepackage{multicol}
\usepackage{a4wide}
\usepackage{latexsym}
\usepackage{verbatim}
\setlength{\textheight}{23.8cm}
\setlength{\topmargin}{-2.5cm}
\setlength{\textwidth}{165mm}
\setlength{\oddsidemargin}{0mm}
\usepackage{eurosym}

%\renewcommand{\baselinestretch}{0.85}

%\input{macroAlgo}
\dontprintsemicolon

\setlength{\parindent}{0pt}  %%suppression indentation

%% ACRONYM DEFINITIONS

\newcommand{\lcx}[1]{\MakeLowercase{#1}}
%\renewcommand{\log}[0]{\text{ln}}
\renewcommand{\iint}[0]{\int\!\!\!\!\!\int}
\renewcommand{\iiint}[0]{\int\!\!\!\!\!\int\!\!\!\!\!\int}
\renewcommand{\iiiint}[0]{\int\!\!\!\!\!\int\!\!\!\!\!\int\!\!\!\!\!\int}
%% expression

%% Latin abrev.
\newcommand{\etc}[0]{\textit{etc} }
\newcommand{\etal}[0]{\textit{et al.} }
\newcommand{\eg}[0]{\textit{e.g.} }
\newcommand{\ie}[0]{\textit{i.e.} }
\newcommand{\cf}[0]{\textit{cf.} }

%% Mathematical coordinates
\newcommand{\meq}[0]{\!\!=\!\!}
\newcommand{\TW}[0]{(\tau, \Omega)}
\newcommand{\tw}[0]{(t, \omega)}
\newcommand{\tv}[0]{(t, \Omega)}
\newcommand{\ts}[0]{(t, s)}
\newcommand{\sw}[0]{(s, \omega)}
\newcommand{\zw}[0]{(z, \omega)}
\newcommand{\nm}[0]{[n,m]}
\newcommand{\uv}[0]{[u,v]}

%% Mathematical constants
\newcommand{\ww}[0]{|\omega|}   %% 1
\newcommand{\www}[0]{|\omega|}  %% []
\newcommand{\te}[0]{T_s}
\newcommand{\fs}[0]{F_s}
\newcommand{\ee}{\,\ensuremath{\mathbf{e}}}

%% Mathematical functions names
\renewcommand{\Re}[0]{\text{Re}}
\renewcommand{\Im}[0]{\text{Im}}
\newcommand{\ri}[0]{\text{Ri}}
\newcommand{\wvd}[0]{\text{WV}}
\newcommand{\cwt}[0]{\text{CW}}
\newcommand{\mwt}[0]{\text{MW}}
\newcommand{\sst}[0]{\text{SST}}
\newcommand{\swt}[0]{\text{SWT}}
\newcommand{\cre}[0]{\text{CRE}}
\newcommand{\card}[1]{\text{Card}(#1)}

%% Mathematical functions names
\newcommand{\rst}[0]{\text{RST}}
\newcommand{\vsst}[0]{\text{VSST}}
\newcommand{\osst}[0]{\text{OSST}}
\newcommand{\lmsst}[0]{\text{LMSST}}

%% Mathematical variables names
\newcommand{\TT}[0]{\mathcal{T}}
\newcommand{\DD}[0]{\mathcal{D}}
%% EOF

\newif\ifcorrection
%\correctiontrue   %% avec correction
\correctionfalse   %% sans correction

\newcommand{\universityname}{IUT d'\'Evry Val d'Essonne}
\newcommand{\deptname}{D\'epartement TC (S1)}
\newcommand{\years}{2023-2024}

\begin{document}
\selectlanguage{francais}
\author{D. Fourer, L. Lagon}
%------------------- TITRE -----------------------------------------
\date{Septembre 2022} 
\TDHead{\universityname}{\deptname}{R1.07, \years}{\large TD2: Taux d'\'evolution}
%-------------------------------------------------------------------


%\underline{Rappels:} 

%Factorielle: $n! = 1\times 2 \times \cdots \times n$\\
%Combinaisons: $C_n^k = \frac{n!}{k! (n-k)!}$\\
%Arrangements: $A_n^k = \frac{n!}{(n-k)!}$\\
%Permutations: $P_n = A_n^n = n!$

\exost Calculer un taux d'\'evolution
  \begin{enumerate}
    \item Une ville de $5\,500$~habitants voit arriver $120$~habitants. Quel est le taux d'\'evolution de sa population?
    \item Le prix d'un objet est de $80$\,\euro. Il diminue de $10$\,\euro. Calculer le taux d'\'evolution de ce prix.
    \item La population d'une ville passe de $28\,040$ à $23\,834$~habitants. Quel est le taux de diminution de cette population?
    \item Un prix passe de $125$\,\euro\ à $140$\,\euro. De quel pourcentage a-t-il augment\'e?
  \end{enumerate}


\exost Calculer une valeur finale
  \begin{enumerate}
    \item Une population de $30\,150$~habitants a augment\'e de $24$\,\%. Calculer la population finale.
    \item Le prix d'un objet est de $235$\,\euro. Il diminue de $30$\,\%. Combien coûte cet objet apr\`es la diminution?
    \item Le prix hors taxes (HT) d'un t\'el\'ephone portable s'\'el\`eve à $85$\,\euro.
    \begin{enumerate}
      \item Calculer le prix toutes taxes comprises (TTC) de ce t\'el\'ephone, sachant que le taux de TVA est de $20$\,\%.
      \item Le vendeur propose une remise de $20$\,\% sur le prix TTC. Calculer le prix effectivement pay\'e par le client.
    \end{enumerate}
    \item En 2010, une île touristique a compt\'e $8\,500$~touristes sur ses terres. L'ann\'ee suivante on a constat\'e une arriv\'ee massive de requins en bordure de mer qui a fait chut\'e le nombre de touristes de $42$\,\%. Combien de touristes sont venus sur cette île en 2011?
  \end{enumerate}


  \exost Calculer une valeur initiale
  \begin{enumerate}
    \item Le prix d'un objet a augment\'e de $25$\,\%. Il coûte à pr\'esent $135$\,\euro. Quel \'etait son prix avant l'augmentation?
    \item Après une r\'eduction de $35$\,\%, un article coûte $79,30$\,\euro. Quel \'etait son prix avant la r\'eduction?
    \item Le prix TTC d'un article est de $385,20$\,\euro\ avec une TVA de $20$\,\%. Quel est son prix hors taxes?
  \end{enumerate}


\exost Calculer un taux global

  \begin{enumerate}
    \item Une action vaut $200$\,\euro. Sur trois mois, son prix augmente de $12$\,\%, puis baisse de $24$\,\% et enfin r\'e-augmente de $12$\,\%. Quel est son prix final? 
    \item Un prix a baiss\'e de $2,4$\,\% par an pendant $8$~ans. D\'eterminer son taux d'\'evolution global.
    \item Un parc d'attraction a accueilli $13$~millions de visiteurs en 2015 et cette fr\'equentation a baiss\'e de $1,8$\,\% en 2016. Quel devrait être le taux d'\'evolution en 2017 pour que le taux global sur la p\'eriode 2015-2017 soit de: \quad (a) $2$\,\%? \quad (b) $-4$\,\%? \quad (c) $0$\,\%?
  \end{enumerate}

\clearpage

\exost Calculer un taux moyen
  \begin{enumerate}
    \item Le prix du baril de p\'etrole a augment\'e de $17$\,\% de juillet 2014 à juillet 2015, et de juillet 2015 à juillet 2016 il a augment\'e de $56$\,\%. D\'eterminer le taux d'\'evolution moyen par an du prix du baril de p\'etrole sur ces deux dernières ann\'ees.
    \item Le prix d'un article vendu, dans une grande enseigne, baisse de $40$\,\% au d\'ebut des soldes puis de $20$\,\% suppl\'ementaires en dernière d\'emarque. Après les soldes, l'enseigne r\'e-augmente le prix de l'article de $50$\,\%. D\'eterminer le taux d'\'evolution moyen du prix de cet article.
    \item Le tableau ci-dessous donne le cours du baril en dollars au mois de juin des ann\'ees 2007 à 2012:
    \begin{center}
     % \begin{tabularx}{0.9\linewidth}{|l|*{6}{>{\centering\arraybackslash$}X<{$}|}}
      \begin{tabular}{|l|l|l|l|l|l|l|}
        \hline
        Ann\'ee                 & 2007  &  2008  & 2009  & 2010  &  2011  & 2012  \\
        \hline
        Cours en juin (en \$) & 71,05 & 132,32 & 68,68 & 74,76 & 114,03 & 95,16 \\
        \hline
      \end{tabular}
    \end{center}
    D\'eterminer le taux d'\'evolution moyen du prix du baril sur cette p\'eriode.
  \end{enumerate}


\exost Calculer un indice

  \begin{enumerate}
    \item Le nombre d'abonnements à internet en Chine \'etait de $111,5$~millions en 2009 et de $126,3$~millions en 2010. D\'eterminer l'indice du nombre d'abonnements en 2010, d'ann\'ee de r\'ef\'erence 2009.
    \item En 2011 les droits d'inscription en première ann\'ee d'une universit\'e \'etaient de $177$\,\euro. L'indice des droits d'inscription en 2011, d'ann\'ee de r\'ef\'erence 2010, \'etait de $101,7$. Calculer les droits d'inscription en première ann\'ee de cette universit\'e en 2010.
    \item Les indices des loyers sont donn\'es dans le tableau suivant:
    \begin{center}
      %\begin{tabularx}{0.8\linewidth}{|X|*{5}{>{\centering\arraybackslash$}X<{$}|}}
      \begin{tabular}{|l|l|l|l|l|l|}
        \hline
        Ann\'ee  & 2008 & 2009 & 2010 & 2011  & 2012  \\
        \hline
        Indice & 97,7 & 99,9 & 100  & 101,5 & 103,9 \\
        \hline
      \end{tabular}
    \end{center}
    D\'eterminer les indices des loyers avec 2009 comme ann\'ee de r\'ef\'erence.
  \end{enumerate}




\end{document}

% End Of File

