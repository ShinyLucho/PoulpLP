\documentclass[a4paper, 9pt]{article}
\usepackage[latin1]{inputenc}
\usepackage[T1]{fontenc}
\usepackage[francais]{babel}
\usepackage{entete}
\usepackage{noitemsep}
\usepackage{euscript} 
\usepackage{amsmath,amssymb,amsfonts,amsthm}
\usepackage{graphicx,graphics,epsfig,subfigure,color}
\usepackage{url}
%\usepackage{algorithm2e}
\usepackage{multicol}
\usepackage{a4wide}
\usepackage{latexsym}
\usepackage{verbatim}
\setlength{\textheight}{24cm}
\setlength{\topmargin}{-0.5cm}
\setlength{\textwidth}{160mm}
\setlength{\oddsidemargin}{1mm}

%\renewcommand{\baselinestretch}{0.85}

%\input{macroAlgo}
%\dontprintsemicolon

\setlength{\parindent}{0pt}  %%suppression indentation


\begin{document}
\selectlanguage{francais}
\author{D. Fourer, L. Lagon}
\newcommand{\universityname}{IUT d'\'Evry Val d'Essonne}
\newcommand{\deptname}{D\'epartement TC (S1)}
\newcommand{\years}{2023-2024}

%------------------- TITRE -----------------------------------------
\date{Septembre 2023} 
\TDHead{\universityname}{\deptname}{R1.13, Ressources et Culture Num\'eriques 1, \years}{\large TD1: Utilisation d'un syst\`eme d'information}
%\TDHead{DUT TC}{}{\large TIC3: Fonctions avanc\'ees d'un tableur}
%-------------------------------------------------------------------
%\underline{Objectifs:} Ma\^itriser un logiciel de tableur

\section{Mise en place de l'environnement de travail} %% 20 min

Pour vous aider dans l'activation de votre compte num\'erique, vous pourrez consulter l'URL (Uniform Resource Locator): \url{https://www.univ-evry.fr/numerique.html}.
%% import / export csv, maitrise des formats Libreoffice/Excel
%% Mise en page dans l'environnement excel

%% import d'un fichier csv:   donnee / import fichier txt
\exost Dans votre espace personnel (lecteur r\'eseau not\'e Z dans cet exemple), cr\'eez l'arborescence: \verb? Z:\\RCN1\TP1 ?
dans laquelle vous d\'eposerez le fichier correspondant au sujet \url{http://fourer.fr/Ens/2324/RCN1/td1.pdf}.
Pour les sujets suivants, vous ajouterez au fur et \`a mesure un sous-dossier \verb? TPx ? dans lequel vous d\'eposerez
les fichiers correspondants.

\exost Activez votre compte UEVE en vous connectant \`a l'adresse suivante: \url{https://moncompte.univ-evry.fr/}
Vous devrez vous munir de votre num\'ero \'etudiant remis au moment de votre inscription et il faudra choisir un
mot de passe s\'ecuris\'e (plus de 8 caract\`eres avec au moins une majuscule et un symbole sp\'ecial).

\exost Une fois votre compte UEVE actif, connectez-vous sur la plateforme E-campus \url{https://ecampus.paris-saclay.fr/}.

%%
\section{Utilisation des services num\'eriques de l'universit\'e}  %% 30 min

\exost Connectez-vous \`a \url{https://edt.univ-evry.fr/} et utilisez votre num\'ero de groupe pour consulter votre emploi-du-temps
\'etudiant au format \verb?IUT_TCJ_Fx1_Gy? dans lequel $x$ correspond \`a votre formation (I:initiale, A:alternance) et $y$ votre num\'ero
de groupe de TD. En utilisant la fonction ``Agenda \'electronique'', vous exporterez l'agenda de votre groupe au format ICS
puis vous synchroniserez votre emploi du temps avec votre agenda num\'erique personnel (eg. google calendar, Teams ou IOS Calendar).

%ou sur \url{https://partage.univ-evry.fr}
\exost Connectez-vous sur \url{https://portal.office.com}  et acc\'edez \`a votre messagerie \'etudiante.
Envoyez un e-mail contenant le sujet ``test'' en demandant un accus\'e de r\'eception depuis votre message \'etudiante vers votre messagerie personnelle.
Connectez-vous ensuite sur votre messagerie personnelle et v\'erifiez que l'exp\'editeur correspond \`a:\\
\verb?numero-etudiant@etud.univ-evry.fr?.
Conservez bien pr\'ecieusement les acc\`es \`a votre messagerie qui vous suivra durant toute votre scolarit\'e \`a l'IUT d'\'Evry.

\exost L'universit\'e fournit gratuitement \`a tous les \'etudiants un acc\`es \`a une licence office365.
Sur la plateforme \url{https://office.com}, t\'el\'echargez l'installateur d'office365. Vous d\'eposerez ce dernier dans votre dossier personnel
\verb? Z:\\RCN1\TP1 ?.


\section{Utilisation efficace d'internet}

\vspace{-0.5cm}
\exost Envoyer un fichier volumineux (sup\'erieur \`a 3 Mo) n'est pas recommand\'e par e-mail car cela risque de saturer votre messagerie
ainsi que celle de votre destinataire. Pour \'eviter cela il est recommand\'e de compresser et d'utiliser un service d'h\'ebergement
interm\'ediaire permettant de communiquer une URL permettant au destinataire de t\'el\'echarger votre fichier. 
\begin{itemize}
 \item Compressez le fichier d'installation d'office365 r\'ecup\'er\'e au format zip
 \item En utilisant le service \url{https://wetransfer.com/} ou \url{dl.free.fr}, convertissez en URL
 le fichier .zip obtenu pr\'ec\'edemment
 \item Envoyez-vous par e-mail l'URL g\'en\'er\'ee par wetransfer. Vous vous assurerez que le lien fonctionne et permet de r\'ecup\'erer votre archive.
 %\item Chiffrez ce fichier via le clic-droit (propri\'et\'es/avanc\'e/)%https://www.ionos.fr/digitalguide/serveur/securite/chiffrer-un-fichier-zip/
\end{itemize}

\clearpage

\exost ChatGPT est un assistant bas\'e sur un mod\`ele de langage d\'evelopp\'e par OpenAI 
bas\'e sur les r\'eseaux de neurones artificiels.
Si ce n'est pas d\'ej\`a fait, cr\'eez-vous un compte sur \url{https://chat.openai.com}. 
Vous pouvez tester le syst\`eme en lui posant les questions de votre choix. 
Pensez \`a v\'erifier les sources de chaque r\'eponse avant toute utilisation.


\exost En utilisant les moteurs de recherche ``Qwant'', ``Bind'', ``Google'' et ``ChatGPT'',  r\'ecup\'erez des informations sur {\bf ``la 
Netiquette''}. Vous ferez une synth\`ese des informations r\'ecup\'er\'ees dans un document contenant du texte que vous ouvrirez
avec Libreoffice writer ou Microsoft Word. Votre document contiendra entre 5 et 10 pages et devra respecter les consignes suivantes:

\begin{itemize}
 \item Contenir une page de garde avec un titre, votre identit\'e, le nom de la discipline ainsi que le logo de l'IUT et de l'universit\'e.
 \item Contenir une table des mati\`ere avec la liste des sections.
 \item Utiliser la police ``Arial'' avec un taille de 12 pt.
 \item Contenir des exemples concrets d'utilisation ou non de la netiquette, avec une analyse des avantages et inconv\'enients de son utilisation.
 \item Le document devra \^etre structur\'e et comporter une introduction, une conclusion, ainsi qu'autant de sections n\'ecessaires pour organiser le contenu.
 (eg. Introduction / Description de la netiquette / Exemples d'utilisation / Conclusion)
 \item Il contiendra au minimum 3 illustrations incorpor\'ees au texte. Chaque figure sera num\'erot\'ee et comportera un titre et une r\'ef\'erence \`a sa source
 si vous n'\^etes pas son auteur. (vous vous assurerez que les figures utilis\'ees sont libres de droits)
\end{itemize}

Vous prendrez soin de citer toutes les sources des informations recueillies dans une section ``r\'ef\'erences'' \`a la fin de votre document.
Chaque source comportera un num\'ero unique (eg. $[1]$) qui sera utilis\'e autant de fois que n\'ecessaire pour argumenter chaque propos.

D\`es la cr\'eation du document, enregistrez celui-ci au format .docx et placez le dans votre dossier personnel, par exemple: \verb? Z:\\RCN1\TP1 ?.
Vous v\'erifierez que votre document s'ouvre correctement avec libreoffice et Microsoft Word.

\exost Exportez le document produit au format PDF.

% \exost Ouvrez le fichier \verb? classe.csv ? avec un \'editeur de texte (e.g. Notepad). \`A quoi ce format
% correspond-il? Quels sont les caract\`eres qui permettent de dissocier chaque champ de valeur?
% 
% \exost Ouvrez le fichier \verb? classe.csv ?  avec le logiciel tableur de votre choix (Libreoffice Calc ou Microsoft Excel)
% et v\'erifiez que le contenu s'affiche correctement. Vous devrez d'abord d\'efinir le format du fichier csv permettant de dissocier 
% chaque cellule (s\'eparateurs).
% 
% \exost Maintenant ouvrez le fichier \verb? coefficients.csv ? et importez son contenu dans une nouvelle feuille de calcul.
% Vous prendrez le soin de donner un nom \`a chaque feuille de calcul (e.g. ``notes \'etudiants'' et ``coefficients des mati\`eres'').
% 
% \exost Effectuez un tri des lignes par ordre alphab\'etique des noms de famille. %(ordre lexicographique croissant)
% Assurez vous que la premi\`ere ligne contenant les titres est inchang\'ee et que chaque nom conserve bien les informations correspondantes.
% Vous pourrez comparer le r\'esultat obtenu avec le contenu du fichier csv d'origine.
% 
% \exost Mettez en page vos feuilles de calcul excel en ins\'erant des lignes de s\'eparations et en agrandissant les titres.
% V\'erifiez le format de chaque cellule et en particulier les champs correspondant aux dates de naissance.
% Figez la premi\`ere ligne contenant les titres des colonnes afin que celle-ci reste visible m\^eme lorsque l'\'ecran d\'efile vers le bas.
% 
% \exost Enregistrez votre travail dans un nouveau fichier \verb? classe.xlsx ?. Vous vous assurerez que le fichier porte bien l'extension
% XLSX et qu'il contient bien l'ensemble des modifications effectu\'ees quand vous l'ouvrez avec Libreoffice Calc ou Microsoft Excel.
% 
% \exost Essayez d'ouvrir le m\^eme fichier avec Libreoffice Calc et avec Microsoft Excel. Remarquez vous des diff\'erences?
% 
% \clearpage
% 
% \section{Formules math\'ematiques} %%duree 30min, 1h maxi
% 
% \exost Ins\'erez une nouvelle colonne avant la colonne \verb? NOM ? que vous nommerez \verb? ID ? (identifiant).
% Affectez un num\'ero distinct \`a chaque \'etudiant. Vous commencerez par affecter un nombre au premier \'etudiant
% puis vous appliquerez la formule $=x+1$ \`a l'\'etudiant suivant ($x$ repr\'esente le code de la cellule associ\'e \`a l'ID pr\'ec\'edent).
% En utilisant la fonction copier-coller, vous appliquerez cette formule \`a l'ensemble des \'etudiants.
% 
% \exost Ins\'erez une nouvelle colonne que vous appellerez \verb? AGE ?. En utilisant la fonction \verb?DATEDIF()?,
% calculez l'age de chaque \'etudiant \`a partir de sa date de naissance. Si vous le souhaitez, vous pourrez utiliser la fonction
% \verb?MAINTENANT()? qui retourne la date du jour.
% 
% \exost Ins\'erez une nouvelle colonne apr\`es la derni\`ere note que vous appellerez \verb? MOYENNE ?. 
% Vous ins\'ererez la moyenne pour l'ensemble des mati\`eres de chaque \'el\`eve dans cette colonne en utilisant 
% la fonction \verb? =MOYENNE(x:y) ? o\`u $x$ et $y$ sont respectivement les codes de la premi\`ere et de la derni\`ere 
% cellule prise en compte dans le calcul.
% 
% %%duree 30min, 1h maxi
% \exost Ajoutez une nouvelle colonne \verb? MEDIANE ? puis calculez la note m\'ediane pour chaque \'etudiant. 
% Vous pourrez rechercher la fonction correspondante en utilisant l'assistant (bouton ``Fx'').
% 
% \exost Ajoutez une nouvelle colonne \verb? MOYENNE PONDEREE ? Pour cela vous impl\'ementerez la formule suivante:
% 
% \begin{equation}
%  \bar{x} = \frac{\sum_i c_i x_i}{\sum_i c_i} = \frac{c_1 x_1 + c_2 x_2 + ...}{c_1 + c_2 + ...}
% \end{equation}
% 
% o\`u les $c_i$ correspondent aux coefficients de chaque mati\`ere que vous r\'ecup\'ererez dans la seconde feuille de calcul et
% les $x_i$ correspondent aux notes obtenues pour chaque mati\`ere $i$.
% 
% \exost Utilisez la fonction \verb? RANG() ? pour afficher le classement de chaque \'etudiant en fonction de sa moyenne pond\'er\'ee
% dans une nouvelle colonne que vous ajouterez.
% 
% 
% \exost Ajoutez une nouvelle ligne en bas de page que vous intitulerez: ``Total des effectifs''.
% Vous calculerez pour l'ensemble des \'etudiants la moyenne, la moyenne pond\'er\'ee et la m\'ediane pour l'ensemble des effectifs.
% Pour cela vous pourrez appliquer des formules matricielles pour l'ensemble des notes.
% 
% %\section{Graphiques et } %%duree 30min, 1h maxi
% 
% \textbf{BONUS:} Ins\'erez un graphique en permettant de visualiser la r\'epartition F/G, l'\^age et la r\'epartition des notes.

\end{document}

% End Of File

