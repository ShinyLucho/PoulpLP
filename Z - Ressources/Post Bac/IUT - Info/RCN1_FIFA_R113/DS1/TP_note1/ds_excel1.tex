\documentclass[a4paper,8pt]{article}
\usepackage[latin1]{inputenc}
\usepackage[T1]{fontenc}
\usepackage[francais]{babel}
\usepackage{entete}
\usepackage{noitemsep}
\usepackage{euscript} 
\usepackage{amsmath,amssymb,amsfonts,amsthm}
\usepackage{graphicx,graphics,epsfig,subfigure,color}
\usepackage{url}
%\usepackage{algorithm2e}
\usepackage{multicol}
\usepackage{a4wide}
\usepackage{latexsym}
\usepackage{verbatim}
\setlength{\textheight}{23.8cm}
\setlength{\topmargin}{-0.5cm}
\setlength{\textwidth}{160mm}
\setlength{\oddsidemargin}{1mm}
%\usepackage[clearempty]{titlesec}

\thispagestyle{empty}
%\renewcommand{\baselinestretch}{0.85}

%\input{macroAlgo}
%\dontprintsemicolon

\setlength{\parindent}{0pt}  %%suppression indentation


\begin{document}
\selectlanguage{francais}
\author{D. Fourer}
\newcommand{\universityname}{IUT d'\'Evry Val d'Essonne}
\newcommand{\deptname}{D\'epartement TC (S3)}
\newcommand{\years}{2021-2022}

%------------------- TITRE -----------------------------------------
\date{Septembre 2021} 
\TDHead{\universityname}{\deptname}{R1.02, \years}{Nom: \hspace{5cm} Pr\'enom: \hspace{5cm} Groupe:\hspace{2cm}} %\large DS1: Utilisation avanc\'ee d'Excel
%\TDHead{DUT TC}{}{\large TIC3: Fonctions avanc\'ees d'un tableur}
%-------------------------------------------------------------------
\vspace{-0.6cm}
\begin{center}
 \textbf{TP not\'e, Dur\'ee: 1h30, documents interdits,\\objects connect\'es interdits.}
\end{center}
\vspace{-0.4cm}
\small
\underline{Consignes:} 
\begin{itemize}
% \item Vous d\'eposerez les fichiers cr\'e\'es \textbf{dans un dossier portant votre nom} en veillant \`a indiquer le \textbf{num\'ero de poste}.
% \item Vous indiquerez pour chaque question le nom du fichier correspondant et une description succinte de ce qui a \'et\'e fait.
% \item Il n'est pas n\'ecessaire d'imprimer vos codes sources, qui devront \^etre propres et bien comment\'es, \textbf{1 point est r\'eserv\'e \`a la pr\'esentation et \`a l'indentation du code}.
 \item {\bf Le sujet devra imp\'erativement \^etre remis en fin de s\'eance. La non remise du sujet engendrera une note de 0/20}. % \textbf{Les copies remises sans le sujet ne seront pas corrig\'ees}.
 %\item Vous d\'eposerez votre fichier ``{\bf \verb?nom_prenom.xlsx?}'' sur E-Campus: \url{https://ecampus.paris-saclay.fr}.
 \item Envoi du travail (3 fichiers portant votre nom) par e-mail \`a {\bf dominique.fourer@univ-evry.fr}.
 \item \textbf{1 point pour la pr\'esentation et le respect des consignes}.
\end{itemize}

\normalsize
%\vspace{-0.4cm}
%\vspace{1cm}

% \exost (2 points)
% 
% \begin{enumerate}
%  \item 
%  \item Ouvrez le fichier \verb?ds1.xlsx? dans lequel vous ajouterez une feuille de calcul \verb?ex1?.
% %  \item Importez le fichier \verb?resultats.csv? dans la nouvelle feuille \verb?ex1? en vous assurant que les donn\'ees soient au bon format.  %% 1
% %  \item Ajoutez une colonne que vous appellerez ``\^age'' que vous calculerez pour chaque individu \`a partir de la date de naissance. %% 1
% \end{enumerate}
%% texte / formulaire
%\underline{Objectif:} Traitement de donn\'ees commerciales et pr\'esentation de r\'esultats sur support num\'erique.\\
Consignes: R\'ecup\'erez le fichier \verb?https://fourer.fr/dsrcn1.zip? et d\'ecompressez son contenu sur votre bureau.
\vspace{-0.3cm}
%% excel
\section{Utilisation d'un tableur}
%
\exost (2 points) Ouvrez-le fichier \verb?clients.xlsx?.
Ajoutez une nouvelle feuille de calcul {\bf ex1} dans laquelle vous cr\'eerez un tableau crois\'e dynamique (TCD) 
permettant de visualiser le nombre de clients en fonction du moyen de prospection utilis\'e (Origine).
Vous ajouterez dans la m\^eme feuille de calcul un graphique en b\^atons permettant de visualiser cette information (en abscisse
l'origine du client et en ordonn\'ee le nombre correspondant de clients).

\exost (2 points) Vous ajouterez une nouvelle feuille de calcul {\bf ex2} dans laquelle vous ajouterez un tableau crois\'e dynamique (TCD)
permettant de visualiser le nombre de clients prospect\'es ainsi que la moyenne du co\^ut de prospection pour chaque commercial.
Vous ajouterez dans la m\^eme feuille de calcul un graphique en b\^atons permettant de visualiser le nombre de clients 
obtenus par chaque commercial.

\exost (2 points) En utilisant la fonction \verb?DATEDIF(date1,date2,"d")? calculez le temps \'ecoul\'e (nombre de jours)
depuis la date du dernier contact. Vous ajouterez une nouvelle colonne dans la feuille de calcul portant le titre ``nombre de jours \'ecoul\'es''.


\exost (5 points) Ajoutez 3 colonnes portant les titres S1, S2 et S3 correspondant
aux valeurs ci-dessous calcul\'ee \`a l'aide de la fonction \verb?SI(condition;resultat_si_vrai;resultat_si_faux)?:
\begin{itemize}
 \item S1: score d'origine valant 3 si ``client'', 0 dans les autres cas.
 \item S2: score de dernier contact : 5 si le dernier contact date de plus 3 mois, 0 dans les autres cas.
 \item S3: score de co\^ut prospection : 5 si le co\^ut d\'epasse 100 euros, 0 dans les autres cas.
\end{itemize}
Enfin, ajoutez une nouvelle colonne ``score'' correspondant \`a la somme des 3 crit\`eres ci-dessus.
\vspace{-0.3cm}
\section{Rapport professionnel}
Pour alimenter le contenu attendu vous pourrez consulter le site internet suivant: \url{https://www.pourpasunrond.fr/fiche-synthetique-entreprise/}.

%% rapport :
\exost (8 points) Cr\'eez un nouveau document texte {\bf \verb? nom_prenom.docx?} dans lequel vous pr\'esenterez les donn\'ees trait\'ees \`a l'exercice pr\'ec\'edent.
Le Document produit fera au minimum 5 pages de contenu et comportera {\bf une page de garde} et {\bf une table des mati\`eres}.
Vous prendrez soin de num\'eroter chaque page.

Le document r\'ealis\'e respectera le plan ci-dessous et portera sur une soci\'et\'e fictive dont vous serez le g\'erant:
%
\begin{enumerate}
 \item Introduction (section libre introductive)
 \item Pr\'esentation synth\'etique de l'entreprise (identit\'e, histoire, activit\'e, organigramme)
 \item Bilan de prospection reposant sur l'analyse sur Tableur
 \item Perspectives et pr\'esentation des futures actions commerciales
\end{enumerate}

%%

% %% power-point
% 
% \section{Pr\'esentation professionnelle}
% 
% \exost (6 points)
% \exost Cr\'eez une nouvelle pr\'esentation {\bf \verb? nom_prenom.pptx?} permettant de faire une synth\`ese du rapport
% et dont le contenu portera sur l'analyse r\'ealis\'ee sur tableur.
% La pr\'esentation reprendra la structure du plan du rapport pr\'ec\'edent et en fera une synth\`ese en 10 diapositives minimum.
% 
% Indications: 
% \begin{itemize}
%  \item Ajoutez une page de garde et un sommaire reprenant le plan du rapport
%  \item Ins\'erez des images ainsi que des transitions permettant
% \end{itemize}
% 


\end{document}

% End Of File

